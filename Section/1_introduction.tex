% 卒業論文では英文アブストは必要ないので削除してください
\section*{Abstruct}
\newpage
\section*{概要}

\newpage
\tableofcontents
\newpage
\section{緒言}
過去50年で気象災害の発生件数は5倍に増加し,気象災害による経済損失は世界全体で一日あたり3億8300万ドルに達する\cite{wmo2021}.損失や被害を最小化するためには国際的な減災対策が必要であり,その一つの手段としての全球気候モデルによる気候変動の将来予測は不可欠である.全球気候モデルは一般に水平方向に13kmの格子解像度を有しており\cite{},雲内部の詳細な現象を原理的に計算することは難しい\cite{gsm}.したがって,格子内部の状態を大域的に表現する方法としてバルク法が用いられる\cite{bulk}.バルク法は格子内部で発生する現象を代替して表現することが期待されるが,解明されていないものも含めたすべての現象に対してこれを行うことは非常に困難であり,そのために未解明のメカニズムに対しては根拠が曖昧なパラメータ調整が行われている\cite{hourdin}.このように,全球気候モデルによる気候予測では精度と信頼性に課題がある.物理法則に基づくバルク法を開発し全球気候モデルの精度・信頼性を向上されるためには,雲内部で発生している未解明の現象の詳細を解明する必要がある.



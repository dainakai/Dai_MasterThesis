\section{結論}
本研究は,乱流中水滴の衝突・併合効率の実験的検証および導出を最終的な目標としており,ホログラフィによる水滴の3次元衝突併合観測を実現するための基礎の確立とその実証を目的として研究を行った.光伝搬計算による像再生は計算量が膨大であり,さらに衝突確率の小ささのために非効率的であるため,その前処理として近接粒子ホログラムの局所特徴から直接近接粒子を抽出する方法を検討し,その原理と応用を示した.

第1章では,雲粒の3次元衝突併合観測の重要性とその波及効果,研究の現状について述べ,本研究とその目的の位置づけを示した.第2章では,本研究で用いるホログラフィと機械学習の原理を説明し,さらに,近接粒子ホログラムのスペクトル分布上縞パターンが粒子近接距離と一意に対応し,縞の向きと間隔から粒子間距離を決定できることを明らかにした.第3章では,縞パターンと粒子間距離の対応を実証する基礎実験,縞パターンを抽出し近接粒子を検出する画像認識モデル,抽出されたホログラムから像再生によって粒子軌跡を取得し衝突併合を確認する手法それぞれの詳細を与えた.第4章では,第3章に示した基礎実験,モデル構築,実験データの像再生による併合観測を遂行し,以下の結果を得た.

\begin{enumerate}
    \item 奥行方向距離を持たない同径粒子ホログラムを実験によって記録し,第2章で明らかにしたスペクトル分布上縞パターンと粒子間距離の対応が,粒子が重複しない限りで成立することを実験によって確認した.
    \item 像再生を行わない近接粒子検出が原理的に可能であること,フーリエ変換が可逆変換であることを利用して近接粒子ホログラム画像から画像認識モデルを学習し,数値生成ホログラムに対する推論結果から実際に近接粒子を検出可能であることを示した.また,モデルの性能評価によって,モデルのさらなる高性能化が必要であることを示した.
    \item 構築したモデルを実験データに対して適用し,近接検出モデルによる推論結果を可視化した.さらに,位相回復ホログラフィによる3次元像再生を行って得た水滴軌跡を確認することで,実験データに対しても実際に水滴併合を観測可能であることを示した.
\end{enumerate}
これらの結果から,本研究で提案した近接水滴検出手法および水滴併合観測手法の実現可能性が示された.また,本研究で明らかになった水滴衝突観測のためのボトルネックは以下の3点である.
\begin{enumerate}
    \item 本論文では奥行方向距離を持たない同径粒子ホログラムの局所特徴に対して縞パターンを定式化したが,粒子が半径差や奥行位置距離をもつ場合の縞パターンへの影響を定量する必要がある.
    \item 衝突併合効率を算出するために,水滴衝突検出を行う画像認識モデルは衝突水滴を漏れなく検出する必要があり,さらに抽出後ユーザが手動で確認する時間を削減するために可能な限り偽陽性を減らす必要がある.すなわち,RecallおよびPrecisionの両方を向上させる必要がある.
    \item 画像認識モデルで抽出した近接水滴から衝突併合の正否を正しく決定するために,像再生した3次元体積から精度良く奥行位置を決定できる手法を検討する必要がある.
\end{enumerate}
以上の課題が解決されれば,一般の形状と配置条件を持つ近接水滴に対して,ほとんどのプロセスが自動化された衝突併合観測システムを実現できる.これによって様々な流れ場や水滴径および数密度条件に対する衝突併合効率の実験的検証が可能となり,その結果は雲解像モデルに組み込まれることで雲粒の成長過程などの雲の微物理過程解明に寄与する.こうして得た知見を気候モデルのサブグリッドパラメータに適用することで,精度と信頼性の高い気象予報や気候変動の予測が実現する.
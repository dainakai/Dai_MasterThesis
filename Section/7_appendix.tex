\appendix
\section*{付録}
\addtocontents{toc}{\vspace{-10pt}}
\addcontentsline{toc}{section}{付録}

\renewcommand{\thesubsection}{\Alph{subsection}}
\subsection{2粒子ホログラムの近似}
\label{sec:appendix_2particle}

\subsection{異径・奥行方向距離を持つ場合の2粒子ホログラム}
\label{sec:appendix_deviation}


\subsubsection{奥行方向距離を持たない異径粒子組についてのメモ}
AEMでは,粒子1が記録されたときの複素振幅を $\psi_1$,粒子2を $\psi_2$として,
\begin{equation}
    I(x,y) = |\psi_1|^2 + |\psi_2|^2 -1
\end{equation}
が成り立つと仮定した.さらに,
\begin{equation}
    \varphi_1(\alpha,\beta) = \mathcal{F}\{|\psi_1|^2\} \approx \cos{(\pi \lambda z_0 \gamma^2)} \tilde{A_1}
\end{equation}
\begin{align}
    \varphi_2(\alpha,\beta) = \mathcal{F}\{|\psi_2|^2\} &\approx \cos{(\pi \lambda z_0 \gamma^2)} \tilde{A_2} \\
    &= \cos{(\pi \lambda z_0 \gamma^2)} \exp{(-2\pi \mathrm{j} (\alpha \Delta \xi + \beta \Delta \eta))} \tilde{A'_2}
\end{align}
が成り立つ.ただし, $A'_2$は,粒子1とことなる粒径を持つ粒子2が,粒子1同様原点中心になるときの形状関数である.つまり,以下の関係を満たす.
\begin{equation}
    A'_2(x-\Delta \xi, y- \Delta \eta ) = A_2
\end{equation}
ここで,粒子1の粒径を $a$ ,粒子2の粒径を $a+\Delta a$ とすると, $\tilde{A_1}$および $\tilde{A_2}$ はそれぞれ
\begin{equation}
    \tilde{A_1}(\alpha,\beta) = a \frac{J_1(2\pi a \gamma)}{\gamma}
\end{equation}
\begin{equation}
    \tilde{A'_2}(\alpha,\beta) = (a+ \Delta a) \frac{J_1(2\pi (a + \Delta a) \gamma)}{\gamma}
\end{equation}
となる. $\tilde{A'_2}$ を $\tilde{A_1}$ によって表現できたら,AEMで論じた以下の式と剰余項の和として扱える.
\begin{align}
    \varphi &= \varphi_1(\alpha,\beta) + \varphi_2(\alpha,\beta) \\
    &= \tilde{A_1} \cos{(\pi \lambda z_0 \gamma^2) \cos{(\pi  (\alpha \Delta \xi + \beta \Delta \eta))}}\exp{(\pi \mathrm{j} (\alpha \Delta \xi + \beta \Delta \eta))} + O
\end{align}

\begin{equation}
    v_{\theta} = \frac{C}{2\pi r} \left( 1- \exp{\left(-\frac{Ar^2}{2\nu}\right)} \right)
\end{equation}
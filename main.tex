\documentclass[a4paper,12pt,dvipdfmx, titlepage]{jsarticle_measlab}
\bibliographystyle{junsrt_KIT} 
\usepackage{layout}
\usepackage{amsmath}
\usepackage[dvipdfmx]{graphicx}
\usepackage{here} %画像位置
\usepackage{bm}
\usepackage{mathtools}
\usepackage{enumitem}
\setlist[description]{
  font={\normalfont}
}
\usepackage{siunitx}
\usepackage{multirow}

\usepackage[nocompress,super]{cite}
\renewcommand\citeform[1]{(#1)} % 引用番号を括弧で囲む
\renewcommand\citepunct{, }
% \usepackage[japanese]{nomencl}
% \makenomenclature

\let\Re\relax
\DeclareMathOperator{\Re}{Re}
\let\Im\relax
\DeclareMathOperator{\Im}{Im}

% キャプションの設定を変更しないまま Subcaption を導入する.
\makeatletter
\let\MYcaption\@makecaption
\makeatother
\usepackage{subcaption}
\captionsetup{compatibility=false}      % 必要に応じて
\makeatletter
\let\@makecaption\MYcaption
\makeatother
% \pagenumbering{Roman}

% \title{\LARGE Title 1}
% \secondtitle{\LARGE Title 2}
% \author{Your Name}
% \date{5}
% \studentid{Your Student Number}
% \studentname{Your Name}
% \teachername{田中 洋介 准 教 授}
% \secondteachername{村田 滋 教 授}
% \studentid{\hspace{19pt}18123054\hspace{19pt}}
% \studentname{\hspace{19pt}中井 大 \hspace{19pt}}
% \teachername{\hspace{19pt}田中 洋介 准 教 授\hspace{19pt}}
% \secondteachername{\hspace{19pt}村田 滋 教 授\hspace{19pt}}

\pagenumbering{Roman} 

\begin{document}

% 卒業論文では英文アブストは必要ないので削除してください
\section*{Abstruct}
\newpage
\section*{概要}

\newpage
\tableofcontents
\newpage
\section{緒言}
過去50年で気象災害の発生件数は5倍に増加し,気象災害による経済損失は世界全体で一日あたり3億8300万ドルに達する\cite{wmo2021}.損失や被害を最小化するためには国際的な減災対策が必要であり,その一つの手段としての全球気候モデルによる気候変動の将来予測は不可欠である.全球気候モデルは一般に水平方向に13kmの格子解像度を有しており\cite{},雲内部の詳細な現象を原理的に計算することは難しい\cite{gsm}.したがって,格子内部の状態を大域的に表現する方法としてバルク法が用いられる\cite{bulk}.バルク法は格子内部で発生する現象を代替して表現することが期待されるが,解明されていないものも含めたすべての現象に対してこれを行うことは非常に困難であり,そのために未解明のメカニズムに対しては根拠が曖昧なパラメータ調整が行われている\cite{hourdin}.このように,全球気候モデルによる気候予測では精度と信頼性に課題がある.物理法則に基づくバルク法を開発し全球気候モデルの精度・信頼性を向上されるためには,雲内部で発生している未解明の現象の詳細を解明する必要がある.



\newpage
\section{理論}
\subsection{インラインホログラフィによる粒子の記録と3次元再生}\label{sec:in-lineHolography}
微小物体が存在する3次元空間を通過したコヒーレントな単一波長平行光波は,弱位相物体(weak phase object)による光路長の変化・屈折・反射や,あるいは不透明物体(opaque object)による部分的な遮光・周囲の回折などの作用によって変化する.この光波がさらに光軸方向に伝搬し,ある地点でイメージセンサなどによって光強度として記録されたとき,そのうち特に物体の存在によって生じる干渉・回折パターンをホログラムと呼ぶ.また,記録したホログラムから3次元物体像を復元する技術はホログラフィ\cite{gabor}を呼ばれる.光源・物体・記録媒体がすべて同一軸上にある光学系によるホログラフィは,Gaborによって発明されたことからGaborホログラフィ(Gabor's holography),あるいはインラインホログラフィ(in-line holography)と呼ばれる.

平面波の光伝搬はコンピュータで計算可能である\cite{kreis}ため,物体が存在する空間を通過して伝搬した光複素振幅を観測できれば,その光を物体が存在していた位置まで逆伝搬することで物体がある位置の複素振幅から物体の分布や形状を測定できる.ホログラムの干渉・回折パターンには,それを記録した地点の複素振幅が含まれる.以下では,コンピュータによる光複素振幅の伝搬計算,インラインホログラフィにおける物体の記録・再生方法について順に示す.

\subsubsection{単一波長平行光の伝搬計算}\label{sec:lightwaveProp}
この節では,ホログラムの記録とコンピュータによるホログラム再生の原理である光伝搬式とその計算について示す.光伝搬計算には,角スペクトル法\cite{goodman}を用いる.角スペクトル法は,光伝搬をフーリエ変換によって畳み込み積分として表現することで,光伝搬計算を高速化する手法である.また,記録される微粒子像のホログラムを理論的に扱うために,後にKirchhoff-Fresnel回折式の近似表現として導かれるFresnel回折積分\cite{kreis}についても示す.まずは,コンピュータでホログラム再生計算を行う際に用いる角スペクトル法について示す.角スペクトル法では,近軸近似が必要ないため非常に高精度かつ高速な計算が可能である.

% Goodman,早崎本による角スペクトル法の説明
波長$\lambda$の単色平行波$\psi(x,y;z)$は,以下のヘルムホルツ方程式に従う.
\begin{equation}
    \label{th:helmholtz}
    \nabla^2 \psi(x,y;z) + \left(\frac{2\pi}{\lambda} \right)^2 \psi(x,y;z) = 0
\end{equation}
$\psi(x,y;z)$の$x,y$に関する二次元フーリエ変換を$U(\alpha,\beta;z)$とすると,これらは以下の関係に従う.
\begin{align}
    \label{th:lightwaveFourier}
    U(\alpha,\beta;z) &= \mathcal{F}\{\psi(x,y;z)\} = \iint_{-\infty}^{\infty} \psi(x,y;z) \exp{\left(-2\pi \mathrm{j} (x\alpha + y\beta)\right)} \mathrm{d}x\mathrm{d}y \\
    \label{th:lightwaveInverseFourier}
    \psi(x,y;z) &= \mathcal{F}^{-1}\{U(\alpha,\beta;z)\} = \iint_{-\infty}^{\infty} U(\alpha,\beta;z) \exp{\left(2\pi \mathrm{j} (x\alpha + y\beta)\right)} \mathrm{d}\alpha\mathrm{d}\beta
\end{align}
ただし,$\alpha,\beta$はそれぞれ$x,y$のフーリエ変換空間の座標であり,$\mathrm{j}$は虚数単位である.式(\ref{th:lightwaveInverseFourier})を式(\ref{th:helmholtz})に代入すると,以下の式が得られる.

\begin{equation}
    \label{th:angularSpectrumequation}
    \frac{\partial^2 U(\alpha,\beta;z)}{\partial z^2} + \left( \frac{2 \pi}{\lambda} \right)^2 \left[ 1 - \left( \lambda \alpha \right)^2 - \left( \lambda \beta \right)^2 \right]U(\alpha,\beta;z) = 0
\end{equation}
式(\ref{th:angularSpectrumequation})の一般解として,以下の式が得られる.
\begin{equation}
    \label{th:angularSpectrumSolution}
    U(\alpha,\beta;z) = U(\alpha,\beta;0) \exp{\left( \frac{\mathrm{j}2\pi z}{\lambda} \sqrt{1-\left( \lambda \alpha \right)^2 - \left( \lambda \beta \right)^2} \right)}
\end{equation}
ここで,以下に定める関数 $G_z(\alpha,\beta)$ を角スペクトル法の伝達関数と呼ぶ.
\begin{equation}
    \label{th:angularSpectrumTransferFunction}
    G_z(\alpha,\beta) = \exp{\left( \frac{\mathrm{j}2\pi z}{\lambda} \sqrt{1-\left( \lambda \alpha \right)^2 - \left( \lambda \beta \right)^2} \right)}
\end{equation}
以上から,例えば$z=0$面から$z=z_0$面までの平行波の光伝搬は,以下のように計算できる.
\begin{equation}
    \label{th:lightProp_angularSpectrum}
    \psi(x,y,z_0) = \mathcal{F}^{-1}\left\{ \mathcal{F}\{\psi(x,y;0)\} \cdot \exp{\left( \frac{\mathrm{j}2\pi z_0}{\lambda} \sqrt{1-\left( \lambda \alpha \right)^2 - \left( \lambda \beta \right)^2} \right)} \right\}
\end{equation}
コンピュータで有限領域に対して離散的な平行波配列に対して上記の光伝搬計算を行う場合,以下の対応を用いる.
\begin{align}
    (x_i,y_j) &=  \left( i \Delta x , j\Delta y \right)\\
    (\alpha_i,\beta_j) &= \left( \frac{i}{L\Delta x} , \frac{j}{L\Delta y} \right)
\end{align}
ただし,$L$は離散フーリエ変換を行う配列の一辺の要素数である.また,$\Delta x$,$\Delta y$はそれぞれ$x$,$y$の離散化幅である.本研究でコンピュータを用いて行うホログラムの記録・再生に係る光伝搬計算は,すべて上記の角スペクトル法を用いる.

次に,Fresnel-Kirchhoff回折式から従うFresnel回折積分による光伝搬計算について示す.これは,後に理論的な粒子のホログラムを取り扱う際に必要となる.$z=0$面の単色平行光$\psi(x,y;0)$と,これが$z = z_0$面まで伝搬した複素振幅$\psi(x,y;z_0)$は,Fresnel-Kirchhoff回折式により以下の関係を満たす\cite{kreis}.
\begin{equation}
    \label{th:fresnel-kirchhoff}
    \psi(x,y;z_0) = \iint_{-\infty}^{\infty} \psi(x',y';0)\frac{z_0\exp{(\mathrm{j}2\pi r' /\lambda})}{\mathrm{j}\lambda r'^2} \mathrm{d}x'\mathrm{d}y'
\end{equation}
ただし,$r'=\sqrt{(x-x')^2+(y-y')^2+z_0^2}$である.ここで,系が以下の条件を満たすと仮定する.
\begin{equation}
    \label{th:fraunhoferCondition}
    \frac{\pi }{\lambda z_0} \left( {x'}_{\mathrm{max}}^2 + {y'}_{\mathrm{max}}^2 \right) \ll 1
\end{equation}
ただし,${x'}_{\mathrm{max}}$,${y'}_{\mathrm{max}}$は,$(x',y')$面の開口部,あるいは物体像座標の最大値である.この仮定のもと,$r'$を以下のように近似する.
\begin{align}
    r' &= \sqrt{(x-x')^2+(y-y')^2+z_0^2} \\
    &= z_0 \sqrt{1 + \frac{(x-x')^2+(y-y')^2}{z_0^2}} \\
    \label{th:approxOfR}
    &\approx z_0 \left( 1 + \frac{(x-x')^2+(y-y')^2}{2z_0^2} \right)
\end{align}
式(\ref{th:approxOfR})を式(\ref{th:fresnel-kirchhoff})に代入すると,以下のようになる\cite{tyler1976}.
\begin{equation}
    \label{th:fresneldiffraction}
    \psi(x,y;z_0) = \frac{\exp{ \left\{\mathrm{j}2\pi z_0 /\lambda \right\}}}{\mathrm{j}\lambda z_0} \iint_{-\infty}^{\infty} \psi(x',y';0)\exp{\left\{ \mathrm{j}\pi \frac{(x-x')^2+(y-y')^2}{z_0\lambda} \right\}} \mathrm{d}x'\mathrm{d}y'
\end{equation}
式(\ref{th:fresneldiffraction})を,Fresnel回折積分と呼ぶ.

\subsubsection{粒子ホログラムの記録}
\ref{sec:lightwaveProp}節で示した光伝搬計算を用いて,インラインホログラフィによる粒子像の記録を行う.インラインホログラフィにおける物体面,記録面の関係を図\ref{fig:in-lineHolography}に示す.ただし,物体面を表現する$(x',y')$座標は,式(\ref{th:fresnel-kirchhoff})のように回折積分の積分変数として記録面の$(x,y)$座標と区別する必要があるときだけ用いることとする.物体面および記録面で座標の単位長さは等しく,デカルト座標系を用いるため,これらを上に述べた理由以外で区別する必要はない.

\begin{figure}[htbp]
    \centering
    \includegraphics[width=0.8\linewidth]{./Figure/2_Theory/inline_holography.pdf}
    \caption{Schematic of in-line holography: collimated light illuminates a particle on the object plane, and appears as a diffraction pattern  on the recording plane. The irradiance of the complex amplitude on the recording plane is recorded as a hologram.}
    \label{fig:in-lineHolography}
\end{figure}

球形の3次元粒子は,ホログラムの記録時は3次元粒子中心に存在する物体面内の2次元円形として扱う.すなわち,$(x_p,y_p,z_p)$中心の半径 $a$の円は,以下の $z=z_p$上の形状関数$A(x,y;z_p)$として表現する.
\begin{equation}
    \label{th:particleShapeFunction}
    A(x,y;z_p) = \left\{
    \begin{aligned}
        &1 \quad \text{for } r_p \leq a \\
        &0 \quad \text{for } r_p > a
    \end{aligned}
    \right.
\end{equation}
ただし, $r_p=\sqrt{(x-x_p)^2+(y-y_p)^2}$である.今後この節では,便利のために$(x_p,y_p) = (0,0)$,$z_p = 0$とする.物体面$z=0$におけるコヒーレントな平行光の振幅を1,位相を0とすると,物体面における光複素振幅$\psi(x,y;0)$は,以下のようになる.
\begin{align}
    \psi(x,y;0) &= T(x,y)\cdot e^{\frac{2\pi \mathrm{j}}{\lambda}\cdot 0} \\
    &= 1 - A(x,y;0)
\end{align}
ここで,$T(x,y) = 1 - A(x,y;0)$は,物体面における透過関数と呼ばれる.これを$z_0$だけ伝搬して記録面の複素振幅$\psi(x,y;z_0)$を計算すると,式(\ref{th:fresneldiffraction}) に示すFresnel回折積分を用いて以下のようになる\cite{vikram}.
\begin{align}
    \label{th:particleComplexAmplitude}
    \psi(x,y;z_0) &= -\exp{\left( \frac{2\pi \mathrm{j}}{\lambda |z_0|} \right)} \left\{  
        1 + \frac{\mathrm{j}}{\lambda |z_0|} \exp{\left[ \frac{\pi \mathrm{j} \left( x^2+y^2 \right)}{\lambda |z_0|}\right] \tilde{A}\left( \frac{x}{\lambda |z_0|}, \frac{y}{\lambda |z_0|} \right) }
     \right\}
\end{align}
ただし, $\tilde{A}$は形状関数のフーリエ変換であり,以下の実数値関数で与えられる.
\begin{equation}
    \label{th:fourierOfA}
    \tilde{A}(\alpha,\beta) = \mathcal{F}\left\{ A \right\} = \pi a^2 \frac{2J_1(2\pi a \gamma)}{2\pi a \gamma}
\end{equation}
ここで, $\gamma = \sqrt{\alpha^2+\beta^2}$である.式(\ref{th:particleComplexAmplitude})では,以下の形式で用いられる.
\begin{equation}
    \label{th:fourierOfA2}
    \tilde{A}\left( \frac{x}{\lambda |z_0|}, \frac{y}{\lambda |z_0|} \right)  =  \pi a^2 \frac{2J_1(2\pi a r/ \lambda |z_0|)}{2\pi a r/ \lambda |z_0|}
\end{equation}
ただし, $r=\sqrt{x^2+y^2}$である.結局,式(\ref{th:particleComplexAmplitude})は以下のようになる.
\begin{equation}
    \label{th:particleComplexAmplitude2}
    \psi(x,y;z_0) = -\exp{\left( \frac{2\pi \mathrm{j}}{\lambda |z_0|} \right)} \left\{  
        1 + \frac{\mathrm{j}}{\lambda |z_0|} \exp{\left[ \frac{\pi \mathrm{j} \left( x^2+y^2 \right)}{\lambda |z_0|}\right] \pi a^2 \frac{2J_1(2\pi a r/ \lambda |z_0|)}{2\pi a r/ \lambda |z_0|} }
     \right\}
\end{equation}

記録面のホログラム$I(x,y)$は,複素振幅の強度の2乗として以下のように計算できる.
\begin{equation}
    \label{th:irradiance}
    I(x,y) = |\psi(x,y;z_0)|^2 = \psi(x,y;z_0)\psi^*(x,y;z_0)
\end{equation}
これに式(\ref{th:particleComplexAmplitude})を代入して,以下を得る\cite{tyler1976}.
\begin{equation}
    \label{th:particleIrradiance}
    I(x,y) = 1 - \frac{2\pi a^2}{\lambda |z_0|} \sin{\left( \frac{\pi r^2}{\lambda |z_0|} \right)} \frac{2J_1(2\pi ar/\lambda |z_0|)}{2\pi ar / \lambda |z_0|} + \left( \frac{\pi a^2}{\lambda |z_0|} \right)^2 \left[ \frac{2J_1(2\pi ar/\lambda |z_0|)}{2\pi ar / \lambda |z_0|} \right]
\end{equation}

\subsubsection{粒子ホログラムの再生}\label{sec:holographicReconstruction}
\ref{sec:lightwaveProp}節で示した光伝搬計算を用いて,インラインホログラフィによる粒子像の再生を行う.ここで,ある平面波 $u(x,y;z)$ の光伝搬を $u*h_z$ のように表現する.光伝搬の演算 $*h_z$ は,その物理的性質から以下を満たす.
\begin{align}
    \label{th:lightPropOperation}
    \left( u * h_{z_1} \right) * h_{z_2} = u * h_{z_1 + z_2}
\end{align}
また,式(\ref{th:angularSpectrumTransferFunction}),式(\ref{th:fresnel-kirchhoff})より,以下を満たす.
\begin{equation}
    \label{th:lightPropOperation2}
    \overline{h_z} = h_{-z}
\end{equation}
形状関数 $A$ を持つ物体のホログラムは以下で表現される\cite{scott1987}.
\begin{align}
    \label{th:abstructParticleHologram}
    I(x,y) &= \left| \left( 1 - A(x,y) \right) * h_{z_0} \right|^2 \\
    &= \left|  1 - A(x,y)  * h_{z_0} \right|^2 \\
    \label{th:abstructParticleHologram2}
    &= 1 - \overline{A(x,y)}*\overline{h_{z_0}} - A(x,y)*h_{z_0} + \left| A(x,y) * h_{z_0} \right|^2
\end{align}
ここで,$1*h_{z_0}=1$ を用いたが,これは物体面における平行光の位相を $-2\pi z_0 /\lambda$ とすることで成立する.これはのちの計算を簡単にするための操作であり,このような位相のずれを導入してもホログラムおよびその再生像には影響しない.また,式(\ref{th:abstructParticleHologram2})の右辺第3項は十分小さいため省略する.ホログラムを $-z_0$ だけ逆伝搬して物体面の複素振幅を計算することで,物体の再生像を得る.
\begin{align}
    \label{th:particleReconstruction}
    \psi_{rec}(x,y;0) &= I(x,y) * h_{-z_0} \\
    &= 1 - \overline{A(x,y)}*\overline{h_{z_0}}*h_{z_0} - A(x,y)*h_{z_0}*h_{z_0} \\
    &= 1 - \overline{A(x,y)} - A(x,y)*h_{2z_0} \\
    \label{th:particleReconstruction2}
    &= \left( 1 - A(x,y) \right) + (1 - A(x,y))*h_{2z_0} - e^{\frac{2\pi \mathrm{j}z_0}{\lambda}}
\end{align}
ただし,$A(x,y)$が実数値関数であることを用いた.式(\ref{th:particleReconstruction2})の第1項は物体光を,第2項は双画像を,第3項は背景光を表す.双画像はホログラム記録時に位相分布が失われることで生じる.双画像はインラインホログラフィでは原理的に除去できず,ノイズとして扱われる.

\subsubsection{ホログラフィによる計測精度のボトルネック}\label{sec:holographyBottleNeck}
前節の最後では,インラインホログラフィにおける像再生において,ホログラム記録時に位相分布が失われることで双画像が生じることを述べた.この節では,ホログラム計測においてその精度や像品質を低下させる典型的な問題である,双画像問題\cite{kats2010}と奥行方向への像伸び問題\cite{meng1995}について述べる.

本研究のように不透明物体を物体面上の形状関数で表現する場合,双画像による影響は再生時の物体輝度値変化として現れる.記録したホログラムを物体面まで逆伝搬して得た再生像スライスをFig. \ref{fig:twinImage:holo}に,その輝度分布を Fig. \ref{fig:twinImage:profile}に示す.Fig. \ref{fig:twinImage:profile}にはさらに,物体面の透過関数プロファイルも併せて示す.透過関数分布では,定義により粒子像の輝度値は0である.しかし,位相情報が失われたGaborホログラムの再生像では,双画像の影響により粒子輝度値が0でなくなる.さらに,双画像のホログラムフリンジも重畳しノイズとして再生像品質を低下させる.
\begin{figure}[htbp]
    \centering
    \begin{subfigure}[t]{0.48\linewidth}
        \includegraphics[width=\linewidth]{./Figure/2_Theory/twin_image/holo.pdf}
        \caption{Reconstructed image of Gabor hologram}
        \label{fig:twinImage:holo}
    \end{subfigure}
    \hfill
    \begin{subfigure}[t]{0.48\linewidth}
        \includegraphics[width=\linewidth]{./Figure/2_Theory/twin_image/profile.pdf}
        \caption{Normalized wave amplitude profile of Gabor hologram and propagated wavefront along the line $y=\SI{0}{\um}$ }
        \label{fig:twinImage:profile}
    \end{subfigure}

    \caption{Example of a reconstructed image slice of Gabor hologram and its wavefront irradiance profile. The hologram is recorded at $z=\SI{80}{\mm}$, and the image slice is reconstructed at $z=\SI{0}{\mm}$. The particle radius is $a=\SI{50}{\um}$, and the recording wavelength is $\lambda = \SI{632.8}{\nm}$. The reconstructed particle image of the Gabor hologram does not have an amplitude of zero.}
    \label{fig:twinImage}
\end{figure}

次に,奥行方向への像伸び問題について述べる.記録ホログラムを再生して粒子座標を取得するとき,一般に粒子の奥行方向位置 $z_0$ は未知である.したがって,$x$-$y$ 投影像における粒子周辺の輝度分布の$z$ 軸プロファイルから $z_0$ を推定する必要がある.式(\ref{th:particleShapeFunction})に示す形状関数 $A$ で定義される粒子の $z$ 軸輝度プロファイルは,以下で与えられる\cite{born1970,tanaka2024}.
\begin{equation}
    \label{th:particleZprofile}
    I_z(z) := \left| \psi(0,0;z) \right|^2 \approx \left[ 1 + \cos{\left( \frac{\pi r_p^2}{\lambda z} \right)} \right]^2 + \sin^2{\left( \frac{\pi r_p^2}{\lambda z} \right)}
\end{equation}
Gaborホログラムの再生像のz軸輝度プロファイルと,式(\ref{th:particleZprofile})によるプロットをFig. \ref{fig:particleZprofile}に示す.
\begin{figure}[htbp]
    \centering
    \includegraphics[width=0.8\linewidth]{./Figure/2_Theory/zprofile.pdf}
    \caption{Normalized $z$-axis irradiance profile of Gabor hologram and theoretical plot of Eq. (\ref{th:particleZprofile}). The hologram is recorded at $z = \SI{0}{\um}$, particle radius is $a=\SI{50}{\um}$, and the recording wavelength is $\lambda = \SI{632.8}{\nm}$. The same system used in Fig. \ref{fig:twinImage} is used for this calculation.}
    \label{fig:particleZprofile}
\end{figure}
式(\ref{th:particleZprofile})に示すプロファイルの,粒子$z$座標$z=0$に対して対象な零点間距離 $\Delta L$ は以下で与えられる\cite{nakatani2019}.
\begin{equation}
    \label{th:elongationLength}
    \Delta L = \frac{4r_p^2}{2(2n-1)\lambda}, \quad (n=1,2,3,\cdots)
\end{equation}
記録時に位相情報を失うインラインホログラフィによる再生では,$n=1$ は理論値を再現するもののそれより大きい $n$ では異なる.特に,$z=0$ における輝度値が双画像問題により0でなくなるため像抽出が困難になる.もし位相分布を復元してホログラムを再生できれば,後に示すように $n=0$ を中心として輝度値が 0 となる領域を抽出することができ,この領域に対する画像処理で粒子座標 $z$ を推定することができる.


双画像の影響は,再生波面から共役像項を除去したり,記録波面の位相分布を再生するなどして除去される.このために,Off-axis法\cite{offaxis}や位相シフト法\cite{phaseshift},Gerchberg-Saxtonアルゴリズムを用いた位相回復法\cite{phaseretrieval}などが用いられる.本研究では,後に示すように2台のカメラを用いた位相回復ホログラフィによってこれに対処する.


\subsection{2粒子ホログラムの局所特徴}\label{sec:twoParticleHologramFeature}
この節では,中心座標が $(x_{p1},y_{p1},z_{p1})$,半径 $a_1$ の粒子1と,中心座標が$(x_{p2},y_{p2},z_{p2})$,半径 $a_2$ の粒子2が近接したときに記録されるホログラフィックパターンの局所特徴について述べる.局所特徴とは画像上のある領域における幾何的な特徴を指す.すなわち,\ref{sec:holographicReconstruction}節で示したホログラムの3次元再生を行わず,ホログラムの画像としての特徴から直接粒子組の近接状態を識別することを目的とする.

\subsubsection{同径かつ奥行方向距離を持たない2粒子のホログラムとそのスペクトル分布}\label{sec:twoParticleHologram}
最も単純な場合として,2粒子が同じ半径 $a$ を持ち,かつ$z$軸方向距離を持たない場合を考える.すなわち,以下の条件を満たす.
\begin{align}
    \sqrt{(x_{p1}-x_{p2})^2+(y_{p1}-y_{p2})^2} > a_1+a_2  \quad &\text{条件1: 点間の距離が半径の和より大きい} \\
    a_1 = a_2 \quad &\text{条件2: 二つの半径が等しい} \\
    z_{p1} = z_{p2} \quad &\text{条件3: 同じz座標上にある}
\end{align}
粒子1の形状関数を $A_1$,粒子2の形状関数を $A_2$ とする.形状関数 $A_i$ は,式(\ref{th:particleShapeFunction})にならって以下のように定義する.
\begin{equation}
    \label{th:eachParticleShapeFunction}
    A_i(x,y;z_{pi}) = \left\{
    \begin{aligned}
        &1 \quad \text{for } \sqrt{(x-x_{p2})^2+(y-y_{p2})^2} \leq a \\
        &0 \quad \text{for } \sqrt{(x-x_{p2})^2+(y-y_{p2})^2} > a
    \end{aligned}
    \right.
\end{equation}
これらの光伝搬による記録ホログラムを計算する.
\begin{align}
    \psi_{z_0} &= \mathcal{F}^{-1}\left\{ \mathcal{F}\{(1-A_1)(1-A_2)\}G(\alpha,\beta;z_{p1}) \right\} \\
    &= \mathcal{F}^{-1}\left\{ \mathcal{F}\{(1-A_1)+(1-A_2)-1\}G(\alpha,\beta;z_{p1}) \right\} \\
    & \begin{multlined} = \mathcal{F}^{-1}\left\{ \mathcal{F}\{1-A_1\}G(\alpha,\beta;z_{p1}) \right\} + \mathcal{F}^{-1}\left\{ \mathcal{F}\{1-A_2\}G(\alpha,\beta;z_{p1}) \right\} \\ - \mathcal{F}^{-1}\left\{ \mathcal{F}\{1\}G(\alpha,\beta;z_{p1}) \right\} \end{multlined}
\end{align}
ここで, $\mathcal{F}^{-1}\left\{ \mathcal{F}\{1-A_i\}G(\alpha,\beta;z_{p1}) \right\}$ は形状関数 $A_i$ を持つそれぞれの粒子が単独で光伝搬した際の複素振幅と等しい.また  $\mathcal{F}^{-1}\left\{ \mathcal{F}\{1\}G(\alpha,\beta;z_{p1}) \right\}$ は背景光の光伝搬を示す.それぞれは式(\ref{th:particleComplexAmplitude})より以下のように計算できる.
\begin{equation}
    \label{th:vikramEq4.8}
    \psi_{pi}(x,y) = -\exp{\left(\mathrm{j}\frac{2\pi}{\lambda}|z_{pi}|\right)} \left\{ 1 + \frac{\mathrm{j}}{\lambda |z_{pi}|} \exp{ \left[ \frac{\mathrm{j} \pi \left( x'^2 + y'^2 \right)}{\lambda |z_{pi}|} \right]} \tilde{A_1} \left(\frac{x'}{\lambda |z_{pi}|},\frac{y'}{\lambda |z_{pi}|} \right)  \right\}
\end{equation}
% vikram 4.8 式
ただし, $x' = x- x_{pi}$, $y' = y-y_{pi}$ である.また,背景光の光伝搬 $\mathcal{F}^{-1}\left\{ \mathcal{F}\{1\}G(\alpha,\beta;z_{p1}) \right\}$ は以下のようになる.
\begin{equation}
    \psi_b(x,y) = \mathcal{F}^{-1}\left\{ \mathcal{F}\{1\}G(\alpha,\beta;z_{p1}) \right\} = \exp{\left(\frac{2\pi \mathrm{j}z_{p1}}{\lambda}\right)}
\end{equation}

さて,今の条件を満たす任意の系は,一方の粒子の中心が $(x,y)$ 平面上の原点に存在するよう座標変換可能である.原点に固定した粒子を粒子1と呼び,他方の粒子を粒子2と呼ぶ.また,粒子2の中心座標は $(\Delta \xi, \Delta \eta)$ とする.すなわち,以下を満たす.
\begin{align}
    \left( x_{p1}, y_{p1} \right) &= (0,0) \\
    \left( x_{p2}, y_{p2} \right) &= (\Delta \xi ,\Delta \eta)
\end{align}
このように定めた粒子1の複素振幅を $\psi_{A_1}$,粒子2の複素振幅を $\psi_{A_1}$ とする.系をFig. \ref{fig:twoParticleHolography}に示す.

\begin{figure}[htbp]
    \centering
    \includegraphics[width=0.8\linewidth]{./Figure/2_Theory/two_particle.pdf}
    \caption{Schematic of two-particle holography: two particles are located at $(0,0)$ and $(\Delta \xi, \Delta \eta)$, respectively. The depth position of the particles are the same, $z_{p1} = z_{p2}$.}
    \label{fig:twoParticleHolography}
\end{figure}

記録されるホログラム $I(x,y)$ は以下のように計算できる.
\begin{align}
    I(x,y) \coloneqq& \left|\psi_{A_1} + \psi_{A_2} - \psi_b \right|^2 \\
    \label{th:extraterms}
    =& |\psi_{A_1}|^2 + |\psi_{A_2}|^2 + 1 + 2\Re \left\{ \psi_{A_1} \overline{\psi_{A_2}} \right\} - 2\Re \left\{\overline{\psi_b}\left( \psi_{A_1} + \psi_{A_2} \right)\right\}
\end{align}
ここで,式(\ref{th:extraterms})右辺第3項以降を以下のように近似する.
\begin{equation}
    1 + 2\Re \left\{ \psi_{A_1} \overline{\psi_{A_2}} \right\} - 2\Re \left\{\overline{\psi_b}\left( \psi_{A_1} + \psi_{A_2} \right)\right\} = -1
\end{equation}
この近似の詳細については,付録\ref{sec:appendix_2particle}で述べる.結局,ホログラムは以下の式で与えられる.
\begin{equation}
    \label{th:2particleHologram}
    I(x,y) = |\psi_{A_1}|^2 + |\psi_{A_2}|^2 - 1
\end{equation}
したがって,同一物体面上同径近接粒子のホログラムは,二重露光ホログラフィで用いられていた以下の計算を適用可能になる\cite{doubleexposure}.

以降では,式(\ref{th:2particleHologram})に示したホログラムをフーリエ変換して得るスペクトル分布に現れる局所特徴について述べる.第一項および2項のフーリエ変換はそれぞれ以下で与えられる\cite{doubleexposure}.
\begin{align}
    \label{th:particle1Fourier}
    \varphi_{A_1}(\alpha,\beta)  = \mathcal{F}\left\{\left| \psi_{A_1} \right|\right\}  &\approx \cos{ \left( \pi \lambda z_0 \gamma^2 \right) }\tilde{A}(\alpha,\beta)  \\
    \label{th:particle2Fourier}
    \varphi_{A_2}(\alpha,\beta) = \mathcal{F}\left\{\left| \psi_{A_2} \right|\right\}   &\approx \cos{ \left( \pi \lambda z_0 \gamma^2 \right) }\tilde{A_2}(\alpha,\beta) \\
    &=\cos{ \left( \pi \lambda z_0 \gamma^2 \right) } \exp{\left( -2\pi \mathrm{j}\left( \alpha \Delta \xi + \beta \Delta \eta \right) \right)} \tilde{A}(\alpha,\beta)
\end{align}
ここで,$\tilde{A}$ は式(\ref{th:fourierOfA})を用いる.第二項のフーリエ変換 $\varphi_{A_2}$ の計算は,フーリエ変換のシフトルールによる.上の近似は,原点を除く点でのみ成立する.式(\ref{th:2particleHologram})第3項のフーリエ変換はデルタ関数であるため,これを省略して結局以下を得る.
\begin{align}
    \varphi(\alpha,\beta) &= \varphi_{A_1}(\alpha,\beta) + \varphi_{A_2}(\alpha,\beta) \\
    &= \tilde{A} \cos{\left( \pi \lambda z_0 \gamma^2 \right)} \cos{\left( \pi \left( \alpha \Delta \xi + \beta \Delta \eta \right) \right)} \exp{\left( \pi \mathrm{j} \left( \alpha \Delta \xi + \beta \Delta \eta \right) \right)}
\end{align}
したがって,スペクトル $S = |\varphi|^2$ は以下のようになる.
\begin{align}
    S(\alpha,\beta) &= \left| \varphi(\alpha,\beta) \right|^2 \\
    \label{th:2particleSpectrum}
    &= \left| \tilde{A} \right|^2 \cos^2{\left( \pi \lambda z_0 \gamma^2 \right)} \cos^2{\left( \pi \left( \alpha \Delta \xi + \beta \Delta \eta \right) \right)}
\end{align}
ここで,式(\ref{th:2particleSpectrum})の各因数の寄与を調べるために,数値生成したホログラムの例,ホログラムのスペクトル分布,同条件の式(\ref{th:2particleSpectrum})のプロットをFig. \ref{fig:2particleSpectrum}に示す.Fig. \ref{fig:2particleSpectrum:a}に示す物体面面像から式(\ref{th:lightProp_angularSpectrum})に示す光伝搬を用いてホログラムFig. \ref{fig:2particleSpectrum:b}を記録する.Fig. \ref{fig:2particleSpectrum:c}に示すスペクトルは,Fig. \ref{fig:2particleSpectrum:b}に対する高速フーリエ変換で得たものである.Fig. \ref{fig:2particleSpectrum:d}に示すスペクトルは,式(\ref{th:2particleSpectrum})を用いて計算したものである.式(\ref{th:2particleSpectrum})の因数 $\left| \tilde{A} \right|^2$ は,スペクトル分布において周期が変化しない同心円パターンとして現れている.これは,式(\ref{th:fourierOfA})に示す通り伝搬距離 $z_0$ に依存しないため,粒子の伝搬距離の変化に対してこのパターンは変化しない.Fig. \ref{fig:2particleSpectrum:d}では同心円パターンとして現れているが,Fig. \ref{fig:2particleSpectrum:c}では矩形領域に対する離散フーリエ変換の影響でパターンが歪んでいる.因数 $\cos^2{\left( \pi \lambda z_0 \gamma^2 \right)}$ は,スペクトル分布で確認できる残りの同心円パターンとして現れており,座標中心から離れるほど周期が短くなる.このパターンは,例に示す $z_0 = \SI{20}{\mm}$ 程度の小さな伝搬距離であれば確認できるが,一般的な記録条件で用いる伝搬距離ではエイリアシングによって確認できなくなる.最後に,因数 $\cos^2{\left( \pi \left( \alpha \Delta \xi + \beta \Delta \eta \right) \right)}$ はスペクトル分布の縞パターンを表す.このパターンは粒子組の平面内距離 $\Delta \xi$,$\Delta \eta$ に依存し,またこれらのみに依存するため,縞パターンの法線ベクトルから平面内距離を一意に決定できる.本節では同径かつ奥行方向距離を持たない2粒子のホログラムについて調べたが,半径が異なる場合,奥行方向距離を持つ場合のホログラフィックパターンについては付録\ref{sec:appendix_deviation}で述べる.

\begin{figure}[htbp]
    \centering
    \begin{subfigure}[b]{0.45\linewidth}
        \includegraphics[width=\linewidth]{./Figure/2_Theory/spectrum_example/a.pdf}
        \caption{Object plane}
        \label{fig:2particleSpectrum:a}
    \end{subfigure}
    \hfill
    \begin{subfigure}[b]{0.45\linewidth}
        \includegraphics[width=\linewidth]{./Figure/2_Theory/spectrum_example/b.pdf}
        \caption{Recorded hologram}
        \label{fig:2particleSpectrum:b}
    \end{subfigure}

    \begin{subfigure}[b]{0.45\linewidth}
        \includegraphics[width=\linewidth]{./Figure/2_Theory/spectrum_example/c.pdf}
        \caption{Computationally calculated spectrum of the hologram Fig. \ref{fig:2particleSpectrum:b}}
        \label{fig:2particleSpectrum:c}
    \end{subfigure}
    \hfill
    \begin{subfigure}[b]{0.45\linewidth}
        \includegraphics[width=\linewidth]{./Figure/2_Theory/spectrum_example/d.pdf}
        \caption{Spectrum of the hologram Fig. \ref{fig:2particleSpectrum:b} calculated by Eq. (\ref{th:2particleSpectrum})}
        \label{fig:2particleSpectrum:d}
    \end{subfigure}

    \caption{Example of a hologram and its spectrum. The radii of particles are $a_1 = a_2 = \SI{50}{\um}$. The propagation distance is $z_0 = \SI{20}{\mm}$. The distance between the centers of the particles is $\Delta \xi = \Delta \eta = \SI{100}{\um}$. The wavelength of the light is $\lambda = \SI{532}{\nm}$. The number of discrete points per side of the rectangular area to be Fourier transformed is 1024, and the pixel pitch is \SI{10}{\um}.} 
    \label{fig:2particleSpectrum}
\end{figure}

\subsubsection{離散スペクトル分布の縞パターンによる平面内距離の決定}
縞パターンの向き・および周期と平面内距離の関係が一意であることを\ref{sec:twoParticleHologram}節で示した.この節では,2粒子ホログラム画像のスペクトル分布上に現れる縞パターンから,粒子組の平面内距離 $(\Delta\xi,\Delta\eta)$ を決定する方法を示す.Fig. \ref{fig:stripePatternVector}に示すように縞パターンの状態を定義する.図中$\alpha$,$\beta$軸に沿って定義される離散点間隔 $l,\,m \mathrm{[pixel]}$ は,各軸の縞間隔である.図中では軸の次元と一致させるためにフーリエ変換を施す矩形領域一辺のピクセル長 $L \mathrm{\,[pixel]}$ とピクセルピッチ $\Delta x \mathrm{\,[\si{\um}]}$ の積で除している.
\begin{figure}[htbp]
    \centering
    \includegraphics[width=0.8\linewidth]{./Figure/2_Theory/stripe_pattern.pdf}
    \caption{Schematic diagram defining the orientation and spacing of the stripe pattern using the displacements $l$ and $m$ along the $\alpha$ and $\beta$ axes. The displacement along each axis determines the characteristics of the stripe pattern and corresponds uniquely to the proximity state of the two particles}
    \label{fig:stripePatternVector}
\end{figure}

離散格子座標 $(x_i,y_i)$,$(\alpha_i.\beta_j)$ は,以下の関係を満たす.
\begin{align}
    \left( x_i, y_j \right) &= \left( i \Delta x, j \Delta x \right) \\
    \left( \alpha_i, \beta_j \right) &= \left( \frac{i}{L\Delta x}, \frac{j}{L\Delta x} \right)
\end{align}
以上の関係と,縞パターンの位相周期が $\pi$ であること,粒子組の接近角度と縞パターンの角度が等しいことから,以下の関係が成り立つ.
\begin{gather}
    \pi \left( \frac{l}{L\Delta x} \Delta \xi + \frac{m}{L\Delta x} \Delta \eta \right) = \pi \\
    \frac{\Delta \eta}{\Delta \xi} = \frac{m}{l}
\end{gather}
したがって,以下の関係を導出できる\cite{aem2023}.
\begin{equation}
    \label{th:stripepattern}
    \Delta \xi = \frac{lL\Delta x}{l^2+m^2}, \hspace{5mm} \Delta \eta = \frac{m L\Delta x}{l^2+m^2}
\end{equation}
$l$ および $m$ は $2 \leq l, m < L$ を満たす必要があるため,測定可能な平面内距離は以下の範囲に限られる.
\begin{equation}
    \Delta x \leq \Delta \xi < \frac{L\Delta x}{4}, \hspace{5mm} \Delta x \leq \Delta \eta < \frac{L\Delta x}{4}
\end{equation}

\subsection{畳み込みニューラルネットワークによる局所特徴識別}\label{sec:convolutionalNeuralNetwork}
\ref{sec:twoParticleHologramFeature}節において,近接粒子組のホログラムをそのスペクトル分布の局所特徴から識別可能であることを示した.本論文では,この原理を用いて,畳み込みニューラルネットワーク(Convolutional Neural Network: CNN)\cite{lecun1998}をベースとした画像認識モデルによって近接粒子組の識別を行う.このために本節では,画像認識モデルの学習・推論の原理と基本的な構造について示す.\ref{sec:supervisedClassification}節から\ref{sec:convolutionalLayer}までは機械学習の代表的なテキストであるBishop (2006)による\cite{bishop2006}.

\subsubsection{教師あり分類モデルの学習・推論}\label{sec:supervisedClassification}
教師あり分類モデルは,与えられたデータ $\bm{x}_i$ に対してそれが属する分類のクラス $C_k$ を推定する.本研究において,データのクラス数 $K$ はつねに2である.一般に深層学習を用いる識別モデルでは,データの正解ラベル $\bm{t}_i$,すなわちモデルが学習の後の推定結果として出力するのが望ましいデータは$K$次元ベクトルであり,例えばクラス1に属するデータの正解ラベルは以下のように表現される.
\begin{equation}
    \bm{t} = \left( 1, 0 \right)^\mathrm{T}
\end{equation}
実際のモデルの出力 $\bm{y}_i$ は,データがクラス1に属する確率 $p(C_1|\bm{x}_i)$ とクラス2に属する確率 $p(C_2|\bm{x}_i)$ を用いて以下のように表現される.
\begin{equation}
    \bm{y}_i = \left( p(C_1|\bm{x}_i), p(C_2|\bm{x}_i) \right)^\mathrm{T}
\end{equation}
$p(C_k|\bm{x}_i)$は,データ $\bm{x}_i$ 与えられた場合にそれがクラス $C_k$ に属する条件付き確率であり,データのクラス$C_k$の事後分布とも呼ばれる.これらは定義により以下の関係を満たす.
\begin{equation}
    \sum_{k=1}^K p(C_k|\bm{x}_i) = 1
\end{equation}
結局,教師あり分類モデルは,データ $\bm{x}_i$ に対してそのラベル $\bm{t}_i$ に十分近い出力 $\bm{y}_i$ を得るための関数 $f$ を決定する問題として定式化できる.
\begin{equation}
    \bm{y}_i = f(\bm{x}_i)
\end{equation}
$f$として3層のロジスティック回帰を用いる.各層$j = 1,2,3$の出力ベクトルを $\bm{\phi}_j$ とすると,各層について以下が成り立つ.
\begin{align}
    \bm{\phi}_1 &= \bm{x}_i \\
    \bm{\phi}_2 &= \sigma(\bm{W}_1 \bm{\phi}_1 + \bm{b}_1) \\
    \bm{\phi}_3 &= \sigma(\bm{W}_2 \bm{\phi}_2 + \bm{b}_2) \\
    \bm{y}_i &= \bm{\phi}_3
\end{align}
$\bm{W}_j$は$\mathrm{dim}\bm{\phi}_{j+1}\times \mathrm{dim}\bm{\phi}_{j}$型の係数行列を表し,$\bm{b}$は各層のバイアスベクトルを表す.$\sigma$はシグモイド関数であり,以下で定義される.
\begin{equation}
    \sigma(x) = \frac{1}{1+\exp{(-x)}}
\end{equation}
このように,入力層 $\bm{\phi}_1$,中間層 $\bm{\phi}_2$,出力層 $\bm{\phi}_3$ からなるモデルを順伝播型ニューラルネットワークと呼ぶ.このようなモデルは,中間層の出力ベクトル $\bm{\phi}_2$の次元を十分大きく取れば $f$と同型の任意の非線形関数を近似することができることが知られている\cite{cybenko1989}.実際の深層学習モデルでは中間層は1層ではなく複数の層からなる場合があり,さらに活性化関数としてシグモイド関数ではなくReLU関数\cite{nair2010}などを用いる場合があるが,これらは計算の効率化や学習の安定化のための工夫であり,本質的には同じモデルである.


このモデルの学習は,データ $\bm{x}_i$ と正解ラベル $\bm{t}_i$ を用いて,以下の損失関数 $E$ を最小化することで行われる.
\begin{equation}
    \label{th:lossFunction}
    E(\bm{W}_1,\bm{W}_2,\bm{b}_1,\bm{b}_2) = -\sum_{i=1}^N \sum_{k=1}^K t_{ik} \ln{y_{ik}}
\end{equation}
上式は交差エントロピー誤差と呼ばれ,二項分類問題における標準的な損失関数である.$N$はデータ数である.この損失関数を最小化するパラメータ $\bm{W}_1,\bm{W}_2,\bm{b}_1,\bm{b}_2$ を求めるために,勾配降下法を用いる.勾配降下法は,損失関数の勾配を用いて,パラメータを以下のように更新する.
\begin{equation}
    \bm{W}_j \leftarrow \bm{W}_j - \eta \frac{\partial E}{\partial \bm{W}_j}, \hspace{5mm} \bm{b}_j \leftarrow \bm{b}_j - \eta \frac{\partial E}{\partial \bm{b}_j}
\end{equation}
上式は更新率$\eta $をパラメータに取る最も単純な勾配降下式であるが,実際の深層学習モデルではAdam\cite{kingma2015}などの高度な勾配降下法が用いられる.勾配降下法によるパラメータ更新を繰り返すことで,損失関数を最小化するパラメータを求めることができる.


\subsubsection{畳み込み層による特徴抽出}\label{sec:convolutionalLayer}
画像などの2次元データに対して,ロジスティック回帰層を重ねて得られる全結合ニューラルネットワークを用いて識別を行うと,所望の精度を得るために必要なパラメータ数が非常に多くなる.画像の局所特徴を効率よく抽出する仕組みとして畳み込み層\cite{lecun1998}が提案され,これを用いた画像認識モデルであるAlexNet\cite{krizhevsky2012}が高い性能を示したことで現在に至るまで画像認識モデルの基本的な構造として用いられている.画像認識モデルでは,入力層に近い層で複数の畳み込み層による処理が行われ,特徴抽出・次元削減が行われた後,全結合層に接続し識別を行う.本節では,畳み込み層の原理と構造について示す.

畳み込み層における処理は,カーネルによる畳み込み計算,活性化関数による非線形化,部分サンプリングで構成される.カーネルはカーネルサイズを $L_{k}$ として $L_k \times L_k$ の実数配列であり,二次元データ $\bm{X}_i$ のすべての可能な $L_k \times L_k$ サイズの部分配列と要素ごとの積和を計算する.これらを並べてできる二次元配列 $\bm{Y}_i$ が出力のあるチャネル部分に相当する.データ $\bm{X}_i$ の一辺のデータ数を $L$ としたとき,チャネル出力 $\bm{Y}_i$ の一辺のデータ数は $L-L_k+1$ となるか,あらかじめデータ $\bm{X}_i$ の境界を$L_k$ だけゼロパディングしておくことで $L$ となる.カーネル演算の後に$\bm{Y}_i$と同型のバイアス配列を加える場合もある.これに活性化関数を適用することで,出力 $\bm{Y}_i$ の各要素は以下のように表現できる.
\begin{equation}
    \bm{Y}_i = \sigma \left( \bm{W} \ast \bm{X}_i + \bm{A} \right)
\end{equation}
ひとつの畳み込み層では一般に複数のカーネルを定義し,それぞれのカーネルによる畳み込み計算を行う.畳み込み層への入力データのチャネル数を $C$,カーネル数を $n$ とすると,入力データ・出力データの関係はFig. \ref{fig:convolutionalLayer}に示すようになる.
\begin{figure}[htbp]
    \centering
    \includegraphics[width=0.8\linewidth]{./Figure/2_Theory/cnn.pdf}
    \caption{Schematic diagram of a convolutional layer. The input data $\bm{X}_i$ is a $C\times L \times L$ array, and the output data $\bm{Y}_i$ is a $C\times n\times (L-L_k+1) \times (L-L_k+1)$ array.}
    \label{fig:convolutionalLayer}
\end{figure}

多くの畳み込みニューラルネットワークでは,畳み込み層のあとにプーリング層を挿入する.プーリング層は,入力データの局所的な特徴を保持しつつデータのサイズを小さくするために用いられる.プーリング層では,入力データの局所的な特徴を保持するために,入力データの局所的な領域に対して最大値や平均値を計算する.これらの演算は,入力データのチャネルごとに行われる.プーリング層の出力データのチャネル数は入力データのチャネル数と同じである.プーリング層の出力データのサイズは,入力データのサイズをプーリングサイズで割った商である.最もよく用いられるプーリング処理としてMax-Pooling\cite{nagi2011}がある.これは,入力データの局所的な領域に対して最大値を計算しその領域の代表値とする処理である.$2\times 2$のMax-Poolingを行う場合,出力データのサイズは入力データの半分になる.


\subsubsection{先進的な画像認識モデルとその構造}
AlexNet\cite{krizhevsky2012}の登場以降,VGG (2015)\cite{simonyan2015},ResNet (2016)\cite{he2016},Vision Transformer (2021)\cite{dosovitskiy2021},EffcientNetV2 (2021)\cite{tan2021}などの代表的な画像認識モデルが提案されている.VGGは畳み込み層を用いた大きなモデルの実装により現在でも多くの物体検出モデルの基幹モデルとして採用されている.ResNetはニューラルネットワークの隣接しない層同士を接続するSkip connectionの導入によって大きく精度を向上させた.Vision TransformerはCNNベースではなくTransformer\cite{vaswani2017}ベースのモデルだが,現在に至るまで画像認識モデルの性能評価の基準として用いられている.EfficientNetV2は再びCNNベースのモデルであり,現在のState-of-the-artなモデルとしてデファクトで用いられている.本研究でも,EffcientNetV2を用いて画像認識を行う.

これらのモデルに共通する特徴として,公開されている大規模画像データベースであるImageNet\cite{deng2009}による事前学習を行い,その後識別したいドメインのデータで学習を行っている点である.ImageNetは,\num{21000}クラス以上,\num{1400}万枚の画像からなるデータベースであり,設計したモデルをImageNetで事前学習することで,畳み込み層のパラメータは像の局所特徴を抽出するように最適化される.事前学習したモデルの畳み込み層部分を切り取りその末尾に全結合層を追加し,識別したいドメインのデータで学習を行うことで,識別したいドメインのデータに特化したモデルを得ることができる.これを転移学習\cite{zhuang2020}という.本研究でも,ImageNetで事前学習したEfficientNetV2を用いて転移学習を行う.モデルの構成,学習データ生成,学習方法などの詳細については\ref{sec:EffNetV2}節で示す.


\subsection{位相回復ホログラフィによる水滴像の3次元再生}
この節では,近接粒子の近接検出を行う画像認識モデルを用いて抽出された近接粒子組ホログラムに対して実際に光伝搬計算による3次元像再生を行うときに用いる位相回復ホログラフィについて説明する.まず,位相回復ホログラフィの原理について示し,次に\ref{sec:holographyBottleNeck}節で示したGaborホログラフィによる再生像品質の低下が改善されることを示す.最後に,改善した粒子輝度値の奥行方向プロファイルから粒子の奥行方向位置を決定する手法について示す.

\subsubsection{位相回復ホログラフィの原理}\label{sec:phaseRetrieval}
位相回復ホログラフィ\cite{phaseretrieval}では,物体面からの伝搬距離が異なる二枚のインラインホログラム,すなわち光強度分布を用いて失われた位相分布を復元する\cite{zhang2003,tanaka2016}.このために,同光軸上の2カメラで記録されたホログラムに対してGerchberg-Saxtonアルゴリズム\cite{gerchberg1972}を適用する.物体面から記録面1までの伝搬距離を $z_1$,さらに記録面1から記録面2までの伝搬距離を $\Delta z$ としたときの物体面と各記録ホログラムの関係をFig. \ref{fig:phaseRetrievalOptics}に示す.
\begin{figure}[htbp]
    \centering
    \includegraphics[width=0.8\linewidth]{./Figure/2_Theory/pralignment.pdf}
    \caption{Optical system for phase retrieval holography. The object plane is located at $z = 0$. The distance between the object plane and the first recording plane is $z_1$, and the distance between the first and second recording planes is $\Delta z$.}
    \label{fig:phaseRetrievalOptics}
\end{figure}
位相分布ホログラフィでは,以下の反復計算によって記録面1における複素振幅 $\psi_1$ を復元する.ここで,各記録面$i=1,2$におけるホログラムを$I_i$,$n$回目の反復計算で得た複素振幅を $\psi_i^n$,位相分布を $\phi_i^n$ とし,これらの初期条件$\psi_1^1$および$\phi_1^1$を以下で定める.
\begin{equation}
    \phi_1^1 = 0,  \quad \psi_1^1 = \sqrt{I_i} \exp{(i\phi_i^0)}
\end{equation}
以下の計算ステップに従い $\phi_1^n$ を得る.


\paragraph{STEP 1:}
\begin{equation}
    \psi_2^n = \psi_1^n * h_{\Delta z}
\end{equation}
\begin{equation}
    \phi_2^{n} = \arctan{\left( \frac{\Im \psi_2^n}{\Re \psi_2^n} \right)}
\end{equation}
$\psi_1^n$を記録面2の位置まで伝搬した複素振幅を$\psi_2^n$とする.$\psi_2^n$を用いて記録面2における位相分布 $\phi_2^n$ を計算する.

\paragraph{STEP 2:}
\begin{equation}
    \psi_2^n = \sqrt{I_2} \exp{(\mathrm{j}\phi_2^n)}
\end{equation}
STEP 1で得た位相分布 $\phi_2^n$ を用いて記録面2における複素振幅 $\psi_2^n$ を計算する.$\psi_2^n$の実振幅を$\sqrt{I_2}$で拘束する.

\paragraph{STEP 3:}
\begin{equation}
    \psi_1^{n+1} = \psi_2^n * h_{-\Delta z}
\end{equation}
\begin{equation}
    \phi_1^{n+1} = \arctan{\left( \frac{\Im \psi_1^{n+1}}{\Re \psi_1^{n+1}} \right)}
\end{equation}
STEP 2で得た複素振幅 $\psi_2^n$ を記録面1の位置まで伝搬した複素振幅を$\psi_1^{n+1}$とする.$\psi_1^{n+1}$を用いて記録面1における位相分布 $\phi_1^{n+1}$ を計算する.

\paragraph{STEP 4:}
\begin{equation}
    \psi_1^{n+1} = \sqrt{I_1} \exp{(\mathrm{j}\phi_1^{n+1})}
\end{equation}
STEP 3で得た位相分布 $\phi_1^{n+1}$ を用いて記録面1における複素振幅 $\psi_1^{n+1}$ を計算する.$\psi_1^{n+1}$の実振幅を$\sqrt{I_1}$で拘束する.

\vspace{10pt}
以上の計算を繰り返すことで,記録面1における複素振幅 $\psi_1$ を得る.ふたつの記録ホログラム間距離$\Delta z$は以下の式から決定する\cite{tanaka2016}.
\begin{equation}
    \Delta z = \frac{z_1}{m}, \quad ( m = 1,2,3,\cdots )
\end{equation}

\subsubsection{位相回復ホログラフィによる再生像品質の改善}
位相回復ホログラフィによって粒子再生像がGaborホログラフィからどのように改善するか示す.まず,Fig. \ref{fig:twinImage}に示す系の位相回復ホログラフィによる再生像とその輝度プロファイルをFig. \ref{fig:twinImagePR}に示す.Fig. \ref{fig:twinImage:holo}に見られた双画像によるホログラムフリンジはほとんど消失し,Fig. \ref{fig:twinImage:profile}に示すように粒子位置での輝度値が0となりもとの粒子形状関数をよく再現していることがわかる.
\begin{figure}[htbp]
    \centering
    \begin{subfigure}[t]{0.48\linewidth}
        \includegraphics[width=\linewidth]{./Figure/2_Theory/PR_twin_image/prrec.pdf}
        \caption{Reconstructed image of hologram using phase retrieval holography.}
        \label{fig:twinImagePR:holo}
    \end{subfigure}
    \hfill
    \begin{subfigure}[t]{0.48\linewidth}
        \includegraphics[width=\linewidth]{./Figure/2_Theory/PR_twin_image/prprofile.pdf}
        \caption{Normalized wave amplitude profile of reconstructed images of phase retrieved hologram and gabor hologram along the line $y=\SI{0}{\um}$ }
        \label{fig:twinImagePR:profile}
    \end{subfigure}

    \caption{Example of a reconstructed image slice of phase retrieved hologram and its wavefront irradiance profile. The hologram is recorded at $z=\SI{80}{\mm}$, and the image slice is reconstructed at $z=\SI{0}{\mm}$. The particle radius is $a=\SI{50}{\um}$, and the recording wavelength is $\lambda = \SI{632.8}{\nm}$. The distance between the two recording planes is $\Delta z = \SI{40}{\mm}$, and the phase retrieving iteration number is $n=10$. The phase retrieved profile well reproduces the original particle shape: the twin image is suppressed, and the irradiance value on the particle is equal to 0 as in the original particle shape.}
    \label{fig:twinImagePR}
\end{figure}

次に,奥行方向への像伸びについて示す.Fig. \ref{fig:particlePRZprofile} にGabor再生,位相回復ホログラフィによる再生,式(\ref{th:particleZprofile})による$z$軸方向正規化輝度プロットを示す.Gabor再生像では粒子位置周辺で輝度値が0にならないため位置検出が困難であったが,位相回復ホログラフィによる再生プロファイルでは粒子位置周辺で輝度値が0となる.式(\ref{th:elongationLength})における$m=3$相当の像伸びを確認できる.この$z$軸輝度プロファイルから,粒子の奥行方向位置を決定する手法について\ref{sec:particleDepth}節で示す.
\begin{figure}[htbp]
    \centering
    \includegraphics[width=0.8\linewidth]{./Figure/2_Theory/przprofile.pdf}
    \caption{Normalized $z$-axis irradiance profile of phase retrieved hologram, Gabor hologram and theoretical plot of Eq. (\ref{th:particleZprofile}). The hologram is recorded at $z = \SI{0}{\um}$, particle radius is $a=\SI{50}{\um}$, and the recording wavelength is $\lambda = \SI{632.8}{\nm}$. The same system used in Fig. \ref{fig:twinImagePR} is used for this calculation.}
    \label{fig:particlePRZprofile}
\end{figure}

\subsubsection{粒子奥行位置決定手法}\label{sec:particleDepth}
Fig. \ref{fig:particlePRZprofile}に示すように,位相回復ホログラフィによる再生像では,粒子の真の奥行位置を含む広い領域で輝度値が0となる.奥行方向位置を一点に決定するために,Tamuraの方法を用いる\cite{memmolo2011}.Tamuraの方法では,以下の評価関数 $C^2(z;x,y)$ を最大化するような$z$を求める.
\begin{equation}
    \label{th:tamura}
    C^2(z;x,y) = \frac{\sigma{\left( \bm{I}|_z \right)}}{\mu{\left( \bm{I}|_z \right)}}
\end{equation}
$\sigma$,$\mu$はそれぞれ標準偏差,平均値を示す.$\bm{I}|_z$は,ある奥行位置$z$における再生像$I_z$のうち,$x$-$y$平面内の粒子中心$(x,y)$近傍の粒子像を含む部分画像を示す.このようにして決定された $z$ の解像度は,ホログラムの奥行方向再生解像度$\Delta z$と等しい.

Fig. \ref{fig:tamuratheory}に,Fig. \ref{fig:particlePRZprofile}に示す位相回復した再生像の粒子中心$z$軸プロファイルと,式(\ref{th:tamura})に示すプロットを示す.$\bm{I}|_z$は,粒子像をちょうど包含する$2r_p \times 2r_p$の領域に対して計算する.$C^2$が最大値を取る$z$を奥行粒子位置として決定できることがわかる.
\begin{figure}[H]
    \centering
    \includegraphics[width=0.8\linewidth]{./Figure/2_Theory/tamuraprof.pdf}
    \caption{Tamura $C^2$ function plot of the reconstructed image of phase retrieved hologram. The hologram is recorded at $z = \SI{0}{\um}$, particle radius is $a=\SI{50}{\um}$, and the recording wavelength is $\lambda = \SI{632.8}{\nm}$. The same system used in Fig. \ref{fig:twinImagePR} is used for this calculation.}
    \label{fig:tamuratheory}
\end{figure}



\newpage
\section{方法}


\subsection{近接2粒子ホログラムの検証実験}
式(\ref{th:stripepattern})に示した関係が正しく成り立ち,この関係によって縞パターンから粒子組の近接状態を正しく決定できることを実験によって検証する.実験装置の概念図をFig. \ref{fig:stripePatternExperiment}に示す.本実験では,直径\SI{250}{\um}の円形ドットが印刷されたガラスプレートを2枚用いて粒子間距離 $(\Delta \xi, \Delta \eta )$ を操作し,それぞれの粒子ホログラムのスペクトル分布から粒子間距離を推定する.Table \ref{table:stripePatternExperiment}に実験条件を示す.$y$ 軸方向の粒子間距離は常に $\Delta \eta = \SI{0}{\um}$ とし,$x$軸方向の粒子間距離 $\Delta \xi$ をマイクロメータで操作する.また,本実験では粒子間の$z$軸方向距離が\num{0}であると仮定する.そのために,接着した2枚のガラスプレートの間には,屈折率 \num{1.51253} の屈折率マッチングオイル (Immersion Oil Type A, Cargille) を充填し,2枚のガラスプレート間で回折が起こらないようにする. 

\begin{figure}[H]
    \centering
    \includegraphics[width=0.8\linewidth]{./Figure/3_Methods/stripe_pattern_experiment.pdf}
    \caption{Schematic diagram of experimental setup. The $x$-axis distance between dots is varied by manipulating one of the two glass plates with dots printed on them with a micrometer. The printing surfaces of the glass plates are fixed facing each other. The space between the plates is filled with refractive index matching oil (Immersion Oil Type A, Cargille, Refractive Index: 1.51253) to prevent diffraction caused by the gap between them.}
    \label{fig:stripePatternExperiment}
\end{figure}

\begin{table}[H]
    \centering
    \caption{Experimental conditions to verify the relationship between the stripe pattern formed on the spectral distribution of particle holograms and the distance between particles.}
    \label{table:stripePatternExperiment}
    \begin{tabular}{lll}
    Quantity                               & Value                                             & Unit     \\ \hline \hline
    Dot diameter $2a$                      & 250                                               & \si{\um} \\ \hline
    Dot distance in $x$-axis $\Delta \xi$  & Every 10 from 0 to 500, every 50 from 500 to 1000 & \si{\um} \\ \hline
    Dot distance in $y$-axis $\Delta \eta$ & 0                                                 & \si{\um} \\ \hline 
    Recorded wavelength $\lambda$          & 632.8                                             & \si{\nm} \\ \hline
    Propagated distance $z_0$              & 250                                               & \si{\mm} \\ \hline
    Pixel pitch of hologram $\Delta x$     & 10                                                & \si{\um}
    \end{tabular}
\end{table}


\subsection{水滴近接検出モデルの学習と推論}\label{sec:EffNetV2}
この節では,\ref{sec:twoParticleHologramFeature}節で示した近接粒子ホログラムのスペクトル分布上に現れる縞パターンを検出するための畳み込みニューラルネットワークモデルを構築・学習および評価するための方法について示す.

\subsubsection{学習データ生成}
\ref{sec:in-lineHolography}節および\ref{sec:twoParticleHologramFeature}節で示したように,粒子ホログラムは数値生成が可能である.本論文ではモデル学習のために数値生成したホログラムデータを用いる.生成するデータの諸条件をTable \ref{table:trainingData}に示す.

\begin{table}[H]
    \centering
    \caption{Conditions for generating hologram data for model training.}
    \label{table:trainingData}
    \begin{tabular}{lll}
    Quantity & Value & Unit \\ \hline \hline
    Hologram image size & $\num{256} \times \num{256}$ & \si{pixel\squared} \\ \hline
    Pixel pitch & \num{10} & \si{\um} \\ \hline
    Mean particle diameter & \num{85} & \si{\um} \\ \hline 
    Standard deviation of particle diameter & \num{23} & \si{\um} \\ \hline
    Particle number & up to 3 in positive condition & - \\
    & up to 8 in negative condition & - \\ \hline
    Generated holograms & \num{1000000} for each condition & - \\ \hline
    Propagated distance & \num{220} to \num{270} & \si{\mm} \\ \hline
    Recorded wavelength & \num{632.8} & \si{\nm} \\ \hline
    \end{tabular}
\end{table}

記録されたホログラムのうち,接近粒子組を含むものを positive 条件,接近粒子組を含まないものを negative 条件として分類する.Positive ホログラムには,接近する2つの粒子と余分な1つの粒子を含む3つまでの粒子が記録される.Negative ホログラムには,接近する粒子組を含まない8つまでの粒子が記録される.すべてのホログラムは,人為的に配置される近接粒子組以外の粒子はすべて互いに離れて配置される.本論文におけるPositiveデータの近接距離しきい値は以下によって定める.
\begin{gather}
    \sqrt{\Delta \xi^2 + \Delta \eta^2} < \SI{200}{\um} \\
    |z_{pi}-z_{pj}| < \SI{6000}{\um} \\
\end{gather}
近接しきい値は,平面方向,奥行方向それぞれで粒子組の半径の和より大きければ原理的にはどの値であってもよいが,これらの値が小さいほど実際の時系列画像に対する推論では連続抽出フレーム数が小さくなる.水滴の相対速度が1フレームあたり\SI{20}{pixel}以上ある場合は水滴衝突に際する形状変化を観測することが難しいため,本実験ではこの値に設定する.また,平均粒径\SI{85}{\um}の粒子の伸び長さは式(\ref{th:elongationLength})より$n=1$として約\SI{5.7}{\mm}あるため,平面内距離条件を満たし,かつ奥行方向に\SI{6}{\mm}以内にある粒子組を近接粒子組とする.

以上の条件に従って生成したデータの例とそのスペクトル分布をFig. \ref{fig:trainingData}に示す.Positi
ve データのスペクトル分布には粒子近接に伴う縞パターンが現れており,Negative データのスペクトル分布には縞パターンは現れないことを確認できる.

数値生成したデータのみによる学習では,ノイズを含む実験データに対するモデルの推論性能が低下する可能性がある.そこで,後に実際に推論を行う実験データの背景ノイズを数値生成したデータに付与する.背景ノイズは,Table \ref{table:backgroundnoisecondition}に示す撮影条件で得た高速度画像を背景除去して得る時間列データから得る.ある背景ノイズデータの輝度値分布をFig. \ref{fig:backgroundnoise}に示す.

\begin{figure}[H]
    \centering
    \begin{subfigure}[t]{0.45\linewidth}
        \includegraphics[width=\linewidth]{./Figure/3_Methods/close1.pdf}
        \caption{Positive hologram}
        \label{fig:trainingData:posiholo}
    \end{subfigure}
    \hfill
    \begin{subfigure}[t]{0.45\linewidth}
        \includegraphics[width=\linewidth]{./Figure/3_Methods/far1.pdf}
        \caption{Negative hologram}
        \label{fig:trainingData:negaholo}
    \end{subfigure}

    \begin{subfigure}[t]{0.45\linewidth}
        \includegraphics[width=\linewidth]{./Figure/3_Methods/fftclose1.pdf}
        \caption{Positive hologram spectrum}
        \label{fig:trainingData:posispec}
    \end{subfigure}
    \hfill
    \begin{subfigure}[t]{0.45\linewidth}
        \includegraphics[width=\linewidth]{./Figure/3_Methods/fftfar1.pdf}
        \caption{Negative hologram spectrum}
        \label{fig:trainingData:negaspec}
    \end{subfigure}

    \caption{Example of generated holograms and their spectrum for training. Holograms are generated with the conditions shown in Table \ref{table:trainingData}, and the hologram spectrum is calculated by Fourier transform, logarithmically converted, and normalized. All data above are contrast-enhanced for visualization.} 
    \label{fig:trainingData}
\end{figure}



\begin{table}[H]
    \centering
    \caption{Conditions for generating background noise data.}
    \label{table:backgroundnoisecondition}
    \begin{tabular}{lll}
    Quantity & Value & Unit \\ \hline \hline
    Image size & $\num{256} \times \num{256}$ & \si{pixel\squared} \\ \hline
    Pixel pitch & \num{10} & \si{\um} \\ \hline
    Particle number & 0 & - \\ \hline
    Recorded wavelength & \num{632.8} & \si{\nm} \\ \hline
    \end{tabular}
\end{table}

\begin{figure}[H]
    \centering
    \includegraphics[width=0.8\linewidth]{./Figure/3_Methods/noisehistogram.pdf}
    \caption{Light intensity distribution of background noise data.}
    \label{fig:backgroundnoise}
\end{figure}

\subsubsection{モデル構築および学習}
この節では,Pytorch\cite{paszke2019}とPytorch Image Models (timm)\cite{rw2019timm}を用いてモデルを構築し,学習を行う方法について示す.Table \ref{table:EffNetV2-S}にEfficientNetV2-Sのモデル構成を示す\cite{tan2021}.また,Table \ref{table:EffNetV2-XL}に,EfficientNetV2-SのスケールモデルであるEfficientNetV2-XLのモデル構成を示す.2つのモデルは各Stageにおける層の数やチャネル数,Flatten層直前のMBConv層の有無などが異なるが,Flatten層への入力パラメータ数は1280で共通であり,同じ取扱いが可能である.図中 Conv2d は\ref{sec:convolutionalLayer}節で示した基本的な畳み込み層を示す.Skip connectionはResNet\cite{he2016}で示されたスキップ接続であり,畳み込み層出力を入力を足したものを層の出力とする.SEは Squeeze-and-Excitation \cite{hu2018} を示し,畳み込み層の各チャネルに重み付けを行うアテンション機構である.MBConv\cite{tan2019}やFused-MBConv\cite{tan2021}は,Expansion ratioによって増加させた畳み込み層のチャネル数を再びSEによってもとの次元に減少させてモデルの表現能力を向上させる.畳み込み層の出力はFlatten層で一次元ベクトル化され,2クラス分類を行う全結合層(FC)に入力される.全結合層は出力次元が100, 10, 2の3層からなり,前の2層でBatch Normalization\cite{ioffe2015}とReLU\cite{nair2010}を適用する.最終層はSigmoid関数を適用する.

本論文ではまずEfficientNetV2-Sモデルによって,これまでに局所特徴を示したホログラムのスペクトル分布を学習するモデルと,スペクトル分布ではなくホログラムから直接学習する2つのモデルを比較する.フーリエ変換は可逆変換であり,スペクトル分布上縞パターンの情報はホログラムフリンジにも含まれているため,モデルがホログラム画像から直接学習することが可能であれば前処理としてのフーリエ変換を省略することができ,計算コストを削減できる.ホログラム画像から直接モデルを学習できることを確認した後,よりパタメータが多く,EfficientNetV2-Sよりも特徴抽出能力が高いEfficientNetV2-XLモデルによって学習を行い,推論精度向上を図る.

\newpage
\begin{table}[H]
    \centering
    \caption{EfficientNetV2-S model architecture; this model architecture is the reference implementation by the authors of the paper and differs from the original implementation described in the paper. The original one has another Fused-MBConv layer between Stage 0 and 1.}
    \label{table:EffNetV2-S}
    \begin{tabular}{lllllll}
    Stage & Operator & Kernel size & Stride & Channels & Layers & Other option \\ \hline \hline
    0 & Conv2d & 3 & 1 & 24 & 2 & Skip connection \\ \hline
    1 & Fused-MBConv & 3 & 2 & 48 & 4 & Expansion ratio 4 \\ \hline
    2 & Fused-MBConv & 3 & 2 & 64 & 4 & Expansion ratio 4 \\ \hline
    3 & MBConv & 3 & 2 & 128 & 6 & Expansion ratio 4, \\ &&&&&& SE ratio 0.25  \\ \hline
    4 & MBConv & 3 & 1 & 160 & 9 & Expansion ratio 6, \\ &&&&&& SE ratio 0.25  \\ \hline
    5 & MBConv & 3 & 2 & 256 & 15 & Expansion ratio 6, \\ &&&&&& SE ratio 0.25  \\ \hline
    6 & Flatten & - & - & 1280 & 1 & - \\ \hline
    7 & FC & - & - & 2 & 3 & - \\ \hline
    \end{tabular}
\end{table}

\begin{table}[H]
    \centering
    \caption{EfficientNetV2-XL model architecture; this model is the scaled-up variant of EfficientNetV2-S. The input parameters of the Flatten layer are 1280, consistent with the small variant.}
    \label{table:EffNetV2-XL}
    \begin{tabular}{lllllll}
    Stage & Operator & Kernel size & Stride & Channels & Layers & Other option \\ \hline \hline
    0 & Conv2d & 3 & 1 & 32 & 4 & Skip connection \\ \hline
    1 & Fused-MBConv & 3 & 2 & 64 & 8 & Expansion ratio 4 \\ \hline
    2 & Fused-MBConv & 3 & 2 & 96 & 8 & Expansion ratio 4 \\ \hline
    3 & MBConv & 3 & 2 & 192 & 16 & Expansion ratio 4, \\ &&&&&& SE ratio 0.25  \\ \hline
    4 & MBConv & 3 & 1 & 256 & 24 & Expansion ratio 6, \\ &&&&&& SE ratio 0.25  \\ \hline
    5 & MBConv & 3 & 2 & 512 & 32 & Expansion ratio 6, \\ &&&&&& SE ratio 0.25  \\ \hline
    6 & MBConv & 3 & 1 & 640 & 8 & Expansion ratio 6, \\ &&&&&& SE ratio 0.25  \\ \hline
    7 & Flatten & - & - & 1280 & 1 & - \\ \hline
    8 & FC & - & - & 2 & 3 & - \\ \hline
    \end{tabular}
\end{table}
\newpage

本研究で用いるEfficientNetV2モデルはImageNet-21k\cite{krizhevsky2012}で事前学習したものを用いるため,モデルのFlatten層までの畳み込み層は転移学習ではパラメータを更新しない.ImageNet-21kにはホログラム画像は含まれていないが,この事前学習によって画像認識モデルの畳み込み層は画像特徴を抽出する能力を持つ.抽出された特徴マップから近接粒子の有無を判定するためにランダムな値で初期化した全結合層を接続しており,この層のパラメータを学習することでモデルは近接粒子を検出できるようになる.

モデルの損失関数は交差エントロピー誤差\cite{bishop2006}を,最適化手法はAdam\cite{kingma2015}を用いる.学習率は初期値\num{6e-5}で,学習率減衰は\num{0.7}を用いて,学習率減衰のタイミングは学習のエポック数が\num{3}の倍数のときに行う.この手法はPytorchでStepLRとして提供される.モデルの学習は,学習データのうち\num{5}\%を検証データとして用いる.学習はGPUを用いて行う.

\subsubsection{モデル評価}\label{sec:modelEvaluation}
学習したモデルの評価方法について示す.Table \ref{table:evalclass}に,モデルの推論結果と実際のラベルの組み合わせによって分類される分類クラスを示す.これらのラベルを用いて,モデルの評価指標としてAccuracy,Precision,Recallを以下に定義する.
\begin{align}
    Accuracy &= \frac{TP + TN}{TP + FP + FN + TN} \\
    Precision &= \frac{TP}{TP + FP} \\
    Recall &= \frac{TP}{TP + FN}
\end{align}
Accuracyは主にデータのPositive-Negative比が1に近い場合によく用いられる評価指標である.PrecisionはPositiveと予測したデータのうち,実際にPositiveであるデータの割合を示す.Recallは実際にPositiveであるデータのうち,Positiveと予測されたデータの割合を示す.PrecisionやRecallはデータのPositive-Negative比が1から大きく異なる場合にも用いることができる.これらの評価指標は,モデルの推論結果から予測ラベルをPositiveとするためのしきい値によって変化する.モデルは,データがPositiveである確度(Confidence)を0から1の実数で表現するため,この範囲の変化するしきい値に対してPrecision,Recallを計算し,しきい値で対応するこれらの値をPrecision-Recall図にプロットしたものをPrecision-Recall曲線と呼ぶ.Precision-Recall曲線の下の面積をAUC (Area Under Curve)と呼び,モデルの推論性能を示す指標として用いる\cite{saito2015}.
\begin{table}[H]
    \centering
    \caption{Predicted class by model inference and actual label.}
    \begin{tabular}{c|c | cc}
        \multicolumn{2}{l|}{\multirow{2}{*}{}}  & \multicolumn{2}{c}{Predicted label} \\ \cline{3-4}
        \multicolumn{2}{l|}{}                   & Positive         & Negative         \\ \hline 
        \multirow{2}{*}{True label} & Positive & TP               & FN               \\ \cline{2-4}
                                    & Negative & FP               & TN              
        \end{tabular}
    \label{table:evalclass}
\end{table}


\subsection{時系列水滴ホログラムの撮影実験}
\subsubsection{実験装置と記録条件}
\ref{sec:EffNetV2}節で構築したモデルを用いて,実験で撮影した時系列水滴ホログラムから近接水滴組を抽出する.この節では,時系列水滴ホログラムの記録方法について示す.実験で用いた水滴噴霧装置と光学系をFig. \ref{fig:dropletSpray}に示す.また,記録条件をTable \ref{table:dropletSprayCondition}に示す.本実験では,2つのネブライザ (EW-KA 30, Panasonic) から噴霧した水滴が衝突する領域を2台の高速度カメラ(FASTCAM Mini UX100 type 800K-M-16G, Photron)で撮影する.観測領域からカメラまでの光伝搬距離を変化させ2枚のホログラムを記録することで\ref{sec:phaseRetrieval}節で示した位相回復を行う.カメラに装着したテレセントリックレンズ(VS-LTC1-130/FS, VS Technology)によって平行光をイメージセンサに記録することができる.観測領域で一様等方性乱流を発生させる8つのファンは,すべて定格の\SI{12}{\V}で駆動する.本来は観測領域で適切に一様等方性乱流場が発生しているか検証する必要があるが,本論文ではその検証は行わず,将来の課題とする.

\begin{figure}[H]
    \centering
    \includegraphics[width=0.8\linewidth]{./Figure/3_Methods/dropletspraysystem.pdf}
    \caption{Schematic diagram of droplet spray device and optical system; expanded and collimated He-Ne laser beam illuminates the droplet spray, and the holograms are recorded by two high-speed cameras with telecentric lenses. Eight DC fans are used to generate a uniformly isotropic turbulent field in the observation area.}
    \label{fig:dropletSpray}
\end{figure}

\begin{table}[H]
    \centering
    \caption{Conditions for recording holograms of droplet spray.}
    \label{table:dropletSprayCondition}
    \begin{tabular}{lll}
    Quantity & Value & Unit \\ \hline \hline
    Hologram image size & $\num{1024} \times \num{1024}$ & \si{pixel\squared} \\ \hline
    Pixel pitch & \num{10} & \si{\um} \\ \hline
    Recorded wavelength & \num{632.8} & \si{\nm} \\ \hline
    Propagated distance $z_1$ & \num{220} & \si{\mm} \\ \hline
    Propagated distance $\Delta z$ & \num{110} & \si{\mm} \\ \hline
    Exposure time & \num{250} & \si{\us} \\ \hline

    \end{tabular}
\end{table}

\subsubsection{2カメラの校正}
Fig. \ref{fig:dropletSpray}に示すシステムで撮影した2枚の同時ホログラムは,観測領域からの伝搬距離が異なるため異なるホログラフィックパターンを記録するが,位相回復ホログラフィによってカメラ1における位相分布を復元するためには2枚のホログラムが異なる$z$座標に対して等しい$x$-$y$平面を記録している必要がある.この節では,撮影した2枚のホログラムを校正する方法について示す.

カメラ画像校正にはバンドルアジャストメント\cite{okamoto2014}を用いる.本実験に際しては,直径\SI{70}{\um}のドットがランダムに印刷されたガラスプレートを2台のカメラで撮影し,それぞれをGabor再生して得た像からPIV\cite{willert1991}と等しいアルゴリズムで2カメラの回転・並進によるずれを計算する.コントラストを上げた2枚の再生画像と計算したベクトル場をFig. \ref{fig:beforebundle}に示す.
\begin{figure}[H]
    \centering
    \includegraphics[width=0.98\linewidth]{./Figure/3_Methods/beforebundle.pdf}
    \caption{Reconstructed two random dot images with planer misalignment and the vector field calculated by PIV algorithm. Two images are contrast-enhanced for visualization.}
    \label{fig:beforebundle}
\end{figure}

取得したベクトル場は2枚の画像の平面内のずれを示すため,これを全域で0にするような画像変換$T$を実現すれば2枚の画像を校正できる.この画像変換$\bm{a}$は,カメラ1の座標を $(x_1,y_1)$,変換後のカメラ2の座標を $(x_2,y_2)$ とすると,以下のように表される.
\begin{equation}
    \left\{ 
    \begin{aligned}
        x_2 &= a_0 + a_1 x_1 + a_2 y_1 + a_3 x_1^2 + a_4 x_1 y_1 + a_5 y_1^2 \\
        y_2 &= a_6 + a_7 x_1 + a_8 y_1 + a_9 x_1^2 + a_{10} x_1 y_1 + a_{11} y_1^2
    \end{aligned}
    \right.
    \label{eq:bundle}
\end{equation}
理想の画像変換によって得る$(x_2,y_2)$と上記の画像変換によって得る$(x_2,y_2)$の差を最小化するようにパラメータ$a_i$に関する正規方程式を解く.実際にはパラメータ最適化はニュートン法を用いて行う.このようにして得られた画像変換をFig. \ref{fig:afterbundle}に示す.
\begin{figure}[H]
    \centering
    \includegraphics[width=0.98\linewidth]{./Figure/3_Methods/afterbundle.pdf}
    \caption{Reconstructed two random dot images after bundle adjustment. Two images are contrast-enhanced for visualization.}
    \label{fig:afterbundle}
\end{figure}
以上の手続きで2枚の画像を全域で\SI{1}{pixel}以下の誤差で校正できることを確認した.
\newpage
\section{結果}
\subsection{近接2粒子ホログラムの検証実験}
この実験で記録したホログラムと,そのフーリエ変換で得たスペクトル分布の例を Fig. \ref{fig:stripePatternExpExample} に示す.あ

\begin{figure}[htbp!]
    \centering
    \begin{subfigure}[c]{0.45\linewidth}
        \includegraphics[width=\linewidth]{./Figure/4_Results/stripe_pattern_experiment/recorded_data/a.pdf}
        \caption{Recorded hologram image}
        \label{fig:stripePatternExpExample:a}
    \end{subfigure}
    \hfill
    \begin{subfigure}[c]{0.45\linewidth}
        \includegraphics[width=\linewidth]{./Figure/4_Results/stripe_pattern_experiment/recorded_data/b.pdf}
        \caption{Cropped hologram image of Fig. \ref{fig:stripePatternExpExample:a}}
        \label{fig:stripePatternExpExample:b}
    \end{subfigure}

    \begin{subfigure}[b]{0.8\linewidth}
        \includegraphics[width=\linewidth]{./Figure/4_Results/stripe_pattern_experiment/recorded_data/c.pdf}
        \caption{Computationally calculated spectrum of the hologram Fig. \ref{fig:stripePatternExpExample:b}}
        \label{fig:stripePatternExpExample:c}
    \end{subfigure}

    \caption{Example of a hologram and its spectrum. } 
    \label{fig:stripePatternExpExample}
\end{figure}

\begin{figure}[htbp!]
    \centering
    \includegraphics[width=0.8\linewidth]{./Figure/4_Results/stripe_pattern_experiment/stripe_pattern_exp_exam_peak.pdf}
    \caption{}
    \label{fig:stripePatternExperiment}
\end{figure}

\begin{figure}[htbp!]
    \centering
    \includegraphics[width=0.8\linewidth]{./Figure/4_Results/stripe_pattern_experiment/stripe_pattern_exp_final_result.pdf}
    \caption{}
    \label{fig:stripePatternExperiment}
\end{figure}



\subsection{水滴近接検出モデルの性能評価および実験データに対する推論}

\begin{figure}[H]
    \centering
    \begin{subfigure}[c]{0.45\linewidth}
        \includegraphics[width=\linewidth]{./Figure/4_Results/training/loss_fft.pdf}
        \caption{Loss improvement of the FFT-preprocessed model}
        \label{fig:stripePatternExpExample:a}
    \end{subfigure}
    \hfill
    \begin{subfigure}[c]{0.45\linewidth}
        \includegraphics[width=\linewidth]{./Figure/4_Results/training/acc_fft.pdf}
        \caption{Accuracy improvement of the FFT-preprocessed model}
        \label{fig:stripePatternExpExample:b}
    \end{subfigure}

    \caption{Training results of the FFT-preprocessed model; the model was trained with the dataset of the FFT-preprocessed holograms. The stripe patterns of the proximity droplet holograms were enhanced by the FFT preprocessing, and they can be seen on the center of the spectrum.} 
    \label{fig:stripePatternExpExample}
\end{figure}

\begin{figure}[H]
    \centering
    \begin{subfigure}[c]{0.45\linewidth}
        \includegraphics[width=\linewidth]{./Figure/4_Results/training/loss_raw.pdf}
        \caption{Loss improvement of the raw-input model}
        \label{fig:stripePatternExpExample:a}
    \end{subfigure}
    \hfill
    \begin{subfigure}[c]{0.45\linewidth}
        \includegraphics[width=\linewidth]{./Figure/4_Results/training/acc_raw.pdf}
        \caption{Accuracy improvement of the raw-input model}
        \label{fig:stripePatternExpExample:b}
    \end{subfigure}

    \caption{Training results of the raw-input model; the model was trained with the dataset of the raw holograms. The model can detect the proximity droplets without the FFT preprocessing because Fourier transform is a reversible operation.} 
    \label{fig:stripePatternExpExample}
\end{figure}

\begin{figure}[H]
    \centering
    \includegraphics[width=0.8\linewidth]{./Figure/4_Results/training/outholo.pdf}
    \caption{Example of the inference result of the raw-input model. The proximity droplets are indicated by the green cross symbol. The shade of red overlay indicates the confidence of the model, and the confidence value on the green cross symbol partial area is \num{0.539}. This output image is contrast-enhanced for visualization.}
    \label{fig:stripePatternVector}
\end{figure}

\begin{figure}[H]
    \centering
    \includegraphics[width=0.8\linewidth]{./Figure/4_Results/training/acc_threshold.pdf}
    \caption{Accuracy value for each threshold value. The threshold value of \num{0.5} is the default value.}
    \label{fig:stripePatternVector}
\end{figure}

\begin{figure}[H]
    \centering
    \includegraphics[width=0.8\linewidth]{./Figure/4_Results/training/precision_threshold.pdf}
    \caption{Precision value for each threshold value. The threshold value of \num{0.5} is the default value.}
    \label{fig:stripePatternVector}
\end{figure}

\begin{figure}[H]
    \centering
    \includegraphics[width=0.8\linewidth]{./Figure/4_Results/training/recall_threshold.pdf}
    \caption{Recall value for each threshold value. The threshold value of \num{0.5} is the default value.}
    \label{fig:stripePatternVector}
\end{figure}

\begin{figure}[H]
    \centering
    \includegraphics[width=0.8\linewidth]{./Figure/4_Results/training/prcurve.pdf}
    \caption{Precision-Recall curve of the raw-input model; each point represents the correspondence between the Recall and Precision values for a given discrimination threshold. Thresholds are taken from 0 to 1 in increments of 0.1.}
    \label{fig:stripePatternVector}
\end{figure}

\subsection{衝突水滴組の軌跡再生}

\begin{figure}[H]
    \centering
    \includegraphics[width=0.8\linewidth]{./Figure/4_Results/exp/visout.pdf}
    \caption{Example of the inference result of the raw-input model, for the experimentally recorded hologram. The confidence value on the red shade area is \num{0.635}. This output image is contrast-enhanced for visualization.}
    \label{fig:expInfResult}
\end{figure}

\begin{figure}[H]
    \centering
    \includegraphics[width=0.8\linewidth]{./Figure/4_Results/exp/imp.pdf}
    \caption{$x$-$y$ projection of the reconstructed volume shown in Fig. \ref{fig:expInfResult}}
    \label{fig:expHoloImp}
\end{figure}

\begin{figure}[H]
    \centering
    \includegraphics[width=0.8\linewidth]{./Figure/4_Results/exp/3dview.png}
    \caption{Three dimensional view of the reconstructed volume shown in Fig. \ref{fig:expInfResult}. There is an approaching droplet pair in the red shaded area shown in Fig. \ref{fig:expInfResult}.}
    \label{fig:3dview}
\end{figure}
\newpage
\section{結論}
本研究は,乱流中水滴の衝突・併合効率の実験的検証および導出を最終的な目標としており,ホログラフィによる水滴の3次元衝突併合観測を実現するための基礎の確立とその実証を目的として研究を行った.光伝搬計算による像再生は計算量が膨大であり,さらに衝突確率の小ささのために非効率的であるため,その前処理として近接粒子ホログラムの局所特徴から直接近接粒子を抽出する方法を検討し,その原理と応用を示した.

第1章では,雲粒の3次元衝突併合観測の重要性とその波及効果,研究の現状について述べ,本研究とその目的の位置づけを示した.第2章では,本研究で用いるホログラフィと機械学習の原理を説明し,さらに,近接粒子ホログラムのスペクトル分布上縞パターンが粒子近接距離と一意に対応し,縞の向きと間隔から粒子間距離を決定できることを明らかにした.第3章では,縞パターンと粒子間距離の対応を実証する基礎実験,縞パターンを抽出し近接粒子を検出する画像認識モデル,抽出されたホログラムから像再生によって粒子軌跡を取得し衝突併合を確認する手法それぞれの詳細を与えた.第4章では,第3章に示した基礎実験,モデル構築,実験データの像再生による併合観測を遂行し,以下の結果を得た.

\begin{enumerate}
    \item 奥行方向距離を持たない同径粒子ホログラムを実験によって記録し,第2章で明らかにしたスペクトル分布上縞パターンと粒子間距離の対応が,粒子が重複しない限りで成立することを実験によって確認した.
    \item 像再生を行わない近接粒子検出が原理的に可能であること,フーリエ変換が可逆変換であることを利用して近接粒子ホログラム画像から画像認識モデルを学習し,数値生成ホログラムに対する推論結果から実際に近接粒子を検出可能であることを示した.また,モデルの性能評価によって,モデルのさらなる高性能化が必要であることを示した.
    \item 構築したモデルを実験データに対して適用し,近接検出モデルによる推論結果を可視化した.さらに,位相回復ホログラフィによる3次元像再生を行って得た水滴軌跡を確認することで,実験データに対しても実際に水滴併合を観測可能であることを示した.
\end{enumerate}
これらの結果から,本研究で提案した近接水滴検出手法および水滴併合観測手法の実現可能性が示された.また,本研究で明らかになった水滴衝突観測のためのボトルネックは以下の3点である.
\begin{enumerate}
    \item 本論文では奥行方向距離を持たない同径粒子ホログラムの局所特徴に対して縞パターンを定式化したが,粒子が半径差や奥行位置距離をもつ場合の縞パターンへの影響を定量する必要がある.
    \item 衝突併合効率を算出するために,水滴衝突検出を行う画像認識モデルは衝突水滴を漏れなく検出する必要があり,さらに抽出後ユーザが手動で確認する時間を削減するために可能な限り偽陽性を減らす必要がある.すなわち,RecallおよびPrecisionの両方を向上させる必要がある.
    \item 画像認識モデルで抽出した近接水滴から衝突併合の正否を正しく決定するために,像再生した3次元体積から精度良く奥行位置を決定できる手法を検討する必要がある.
\end{enumerate}
以上の課題が解決されれば,一般の形状と配置条件を持つ近接水滴に対して,ほとんどのプロセスが自動化された衝突併合観測システムを実現できる.これによって様々な流れ場や水滴径および数密度条件に対する衝突併合効率の実験的検証が可能となり,その結果は雲解像モデルに組み込まれることで雲粒の成長過程などの雲の微物理過程解明に寄与する.こうして得た知見を気候モデルのサブグリッドパラメータに適用することで,精度と信頼性の高い気象予報や気候変動の予測が実現する.
\newpage
\section*{謝辞}
\addcontentsline{toc}{section}{\gt 謝辞}
本論文を執筆するにあたり,毎週の報告会や学会発表,論文執筆に際して多くのご指導を賜りました,京都工芸繊維大学計測システム工学研究室准教授 田中洋介 博士に厚く御礼申し上げます.また,同教授 村田茂 博士には,毎月の報告会で近接粒子ホログラムの理論式や実験結果について繰り返し議論していただきました.深く感謝申し上げます.

実験装置の製作にあたり,京都工芸繊維大学オープンファシリティーセンターものづくりユニット 阪本大夢 技術職員にご協力いただきました.感謝申し上げます.

研究室配属から現在に至るまで,研究遂行にあたり互いに協力しあい議論を重ねてきた同研究室同学年の 油谷俊也 さん,北尾優太 さん,来代勝胤 さん,田中智也 さん,田中大貴 さん,中島栞里 さん,濱戸珠樹 さん,松井洸太郎 さんに感謝申し上げます.

また,同様のホログラフィ計測を研究し毎週の報告会で情報共有する中で重要な先行研究の発見や研究手法の着想にご貢献いただいた,同研究室の 石原匠馬 さん,石山満喜 さんにも感謝申し上げます.

最後に,私の研究生活を支えてくれた家族や友人に改めて感謝を申し上げます.
\newpage
\section*{学術論文}
\addtocontents{toc}{\vspace{-10pt}}
\addcontentsline{toc}{section}{\gt 学術論文}

\begin{description}
    \item[(Paper1)] Nakai, D., and Tanaka, Y., “In-Line Digital Holographic Reconstruction by Using GPU Programming with Python”, Advanced Experimental Mechanics, 7, pp.169-173, (2022).
    \item[(Paper2)] Nakai, D., and Tanaka, Y., “Spatial Frequency Analysis of Holograms of Two Droplets in Close Proximity”, Advanced Experimental Mechanics, 8, pp.33-38, (2023).
    \item[(Paper3)] Tanaka, Y., and Nakai, D., "Particle Size Measurement Using a Phase Retrieval Holography System with a GPU-Equipped SBC", KONA Powder and Particle Journal, 41, pp.221-228, (2024).
\end{description}
\section*{受賞}
\addtocontents{toc}{\vspace{-10pt}}
\addcontentsline{toc}{section}{\gt 受賞}
\begin{description}
    \item[(Award1)] 中井大,ベストプレゼンテーション賞,“微小液滴ホログラムのためのCNNベース衝突検知モデルの評価”,可視化情報学会 第50回可視化情報シンポジウム, 東京 工学院大学, 2021/8.
    \item[(Award2)] Nakai, D., Outstanding Oral Presentation Award, “Effect of Depth-Position Displacement of Approaching Microdroplet Pairs on Holographic Patterns”, International Workshop on Advanced Experimental Mechanics for Students and Young Researchers 2022, A050, Online, Nov. 2022.
    \item[(Award3)] 中井大,学長表彰年間賞(学術研究活動),令和4年度 京都工芸繊維大学 学生表彰,京都,京都工芸繊維大学,2023/3.
\end{description}

\section*{研究助成}
\addtocontents{toc}{\vspace{-10pt}}
\addcontentsline{toc}{section}{\gt 研究助成}

\begin{enumerate}
    \renewcommand{\labelenumi}{(\arabic{enumi})}
    \item 独立行政法人日本学術振興会,特別研究員-DC1, “ホログラフィック降水粒子観測システムを用いた乱流場における粒子併合促進効果の解明”,2024年度採用
\end{enumerate}

\section*{国際学術会議}
\addtocontents{toc}{\vspace{-10pt}}
\addcontentsline{toc}{section}{\gt 国際学術会議}
\begin{enumerate}
\renewcommand{\labelenumi}{(\arabic{enumi})}
    \item Nakai, D., and Tanaka, Y., “Study on a Holographic Pattern and its Spectral Distribution Formed by Two Approaching Spheres”, The 32nd International Symposium on Imaging, Sensing, and Optical Memory, C000607, Sapporo, Aug. 1, 2022.
    \item Nakai, D., and Tanaka, Y., “Effect of Depth-Position Displacement of Approaching Microdroplet Pairs on Holographic Patterns”, International Workshop on Advanced Experimental Mechanics for Students and Young Researchers 2022, A050, Online, Nov. 25, 2022.
    \item Nakai, D., and Tanaka, Y., “Measurement of Nanoliter Droplets in a Microchannel Using Phase Retrieval Holography”, The 11th International Conference on Multiphase Flow, \#790, Kobe, Apr. 5, 2023.
    \item Nakai, D., and Tanaka, Y., “Holographic Collision Analysis of Microdroplets: Data Augmentation with OpenFOAM”, ASME-JSME-KSME Fluids Engineering Division 2023, \#3-09-1-04, Osaka, Jul. 12, 2023.
    \item Nakai, D., and Tanaka, Y., “Recognition of Approaching Droplet Holograms using Image Recognition Models”, International Workshop on Advanced Experimental Mechanics for Students and Young Researchers 2023, A042, Osaka, Nov. 25, 2023.
\end{enumerate}

\section*{国内学術会議}
\addtocontents{toc}{\vspace{-10pt}}
\addcontentsline{toc}{section}{\gt 国内学術会議}
\begin{description}
    \item[(O1)] 中井 大, 田中 洋介, “微小液滴ホログラムのためのCNNベース衝突検知モデルの評価”, 可視化情報学会 第50回可視化情報シンポジウム, 東京 工学院大学, 2021/8. 発表日:2022年8月8日
    \item[(O2)] 中井 大, 田中 洋介, “インライン位相回復ホログラフィによるボールレンズの形状計測”, 日本実験力学会 2022年度年次講演会, 鳥取 鳥取大学, 2021/08. 発表日:2022年8月24日
    \item[(O3)] 中井 大, 田中 洋介, “液滴衝突における2液滴間距離がホログラフィックパターンに与える影響”, 日本光学会年次学術講演会 Optics \& Photonics Japan 2022, 栃木 宇都宮大学, 2022/11. 発表日:2022年11月13日
    \item[(P4)] 中井 大, 田中 洋介, “機械学習を用いたマイクロ液滴近接検出”, 日本機械学会関西支部 第98期定時総会講演会, 京都 京都工芸繊維大学, 2023/03. 発表日:2023年3月16日
    \item[(O5)] 中井 大, 田中 洋介, “Julia × GPU によるホログラフィック微粒子計測”, 日本実験力学会 光学的手法分科会研究会, つくば 産総研つくばセンター, 2023/07. 発表日:2023年7月4日
    \item[(O6)] 中井 大, 田中 洋介, “奥行方向距離を持つ近接液滴組のホログラフィックパターンとそのスペクトル分布に関する実験的研究”, 日本実験力学会 2023年度年次講演会, 和歌山 和歌山城ホール, 2023/08. 発表日:2023年8月29日
    \item[(O7)] 中井 大, 田中 洋介, “GPU搭載シングルボードコンピュータと2台のカメラによるリアルタイム位相回復ホログラフィ再生処理モジュールの開発”, 高速度イメージングとフォトニクスに関する総合シンポジウム2023, 東大阪 近畿大学, 2023/12. 発表日:2023年12月14日
\end{description}
\newpage
\bibliography{refs}
\addtocontents{toc}{\vspace{-10pt}}
\addcontentsline{toc}{section}{\gt 参考文献}
\newpage
\appendix
\numberwithin{equation}{subsection}
\numberwithin{figure}{subsection}
\numberwithin{table}{subsection}
\section*{付録}
\addtocontents{toc}{\vspace{-10pt}}
\addcontentsline{toc}{section}{付録}

\renewcommand{\thesubsection}{\Alph{subsection}}
\subsection{2粒子ホログラムの近似}
\label{sec:appendix_2particle}
奥行方向距離を持たない近接同径粒子のホログラフィックパターン導出に際して,以下を仮定した.
\begin{equation}
    1 + 2\Re \left\{ \psi_{A_1} \overline{\psi_{A_2}} \right\} - 2\Re \left\{\overline{\psi_b}\left( \psi_{A_1} + \psi_{A_2} \right)\right\} = -1
    \tag{\ref{th:extratermsApprox}}
\end{equation}
この仮定が,2つの粒子が重複しない条件で妥当であることを示す.

まず,式(\ref{th:extratermsApprox})中の各複素振幅$\psi_{A_1}$,$\psi_{A_2}$,$\psi_{b}$は以下で与えられる.
\begin{align}
    \psi_{A_1}(x,y) &= \exp{\left(\mathrm{j}\frac{2\pi}{\lambda}z_0\right)} \left\{ 1 + \frac{\mathrm{j}}{\lambda z_0} \exp{ \left( \frac{\mathrm{j} \pi r^2}{\lambda z_0} \right)} \pi a^2 \frac{2J_1(2\pi a r/ \lambda z_0)}{2\pi a r/ \lambda z_0}  \right\} \\
    \psi_{A_2}(x,y) &= \exp{\left(\mathrm{j}\frac{2\pi}{\lambda}z_0\right)} \left\{ 1 + \frac{\mathrm{j}}{\lambda z_0} \exp{ \left( \frac{\mathrm{j} \pi r'^2}{\lambda z_0} \right)} \pi a^2 \frac{2J_1(2\pi a r'/ \lambda z_0)}{2\pi a r'/ \lambda z_0}  \right\} \\
    \psi_b(x,y) &= \exp{\left(\mathrm{j}\frac{2\pi}{\lambda}z_0\right)}
\end{align}
ただし,$r = \sqrt{x^2 + y^2}$,$r' = \sqrt{(x-\Delta \xi)^2 + (y-\Delta \eta)^2}$である.また,$z_0>0$とした.このとき,以下が成り立つ.
\begin{align}
    &1 + 2\Re \left\{ \psi_{A_1} \overline{\psi_{A_2}} \right\} - 2\Re \left\{\overline{\psi_b}\left( \psi_{A_1} + \psi_{A_2} \right)\right\} \\
    =& -1 + \frac{1}{\lambda^2 z_0^2} \cos{\left( \frac{\pi \left( r^2-r'^2 \right)}{\lambda z_0} \right) \pi^2 a^4  \frac{2J_1(2\pi a r/ \lambda z_0)}{2\pi a r/ \lambda z_0} \frac{2J_1(2\pi a r'/ \lambda z_0)}{2\pi a r'/ \lambda z_0}}
    \label{appendixEq:extraterm}
\end{align}
式(\ref{appendixEq:extraterm})の右辺第2項が十分小さければ,式(\ref{th:extratermsApprox})は成り立つ.$\Delta \xi =0$,$\Delta \eta=0$であれば,式(\ref{th:particleIrradiance})の右辺第3項と一致することに注意する.

関数$2J_1(x)/x$は$x=0$以外の点で定義されており,常に$|2J_1(x)/x| < 1$である.同様に$|\cos(x)|<1$であるから,式(\ref{appendixEq:extraterm})の右辺第2項は以下を満たす.
\begin{equation}
    -\frac{\pi^2 a^4}{\lambda^2 z_0^2} < \frac{1}{\lambda^2 z_0^2} \cos{\left( \frac{\pi \left( r^2-r'^2 \right)}{\lambda z_0} \right) \pi^2 a^4  \frac{2J_1(2\pi a r/ \lambda z_0)}{2\pi a r/ \lambda z_0} \frac{2J_1(2\pi a r'/ \lambda z_0)}{2\pi a r'/ \lambda z_0}} < \frac{\pi^2 a^4}{\lambda^2 z_0^2}
\end{equation}
無次元数$\pi^2 a^4 / \lambda^2 z_0^2$は半径$a$,粒子の伝搬距離$z_0$に顕著に依存する.ホログラム画像は\SI{8}{bit}で記録され,背景光の実振幅が画像輝度値で64程度になるようカメラの露光時間を調整するため,例えば$64\pi^2 a^4 / \lambda^2 z_0^2$が1より小さい場合は,この成分による輝度値変動はないと言える.本論文の実験条件では,粒子の平均直径が\SI{85}{\um},伝搬距離の最小値が\SI{220}{\mm},記録波長が$\lambda = \SI{632.8}{\nm}$であるから,
\begin{equation}
    \frac{64\pi^2 a^4}{\lambda^2 z_0^2} \approx 0.106
\end{equation}
となり,ホログラムの輝度値変化として記録されない.したがって,式(\ref{th:extratermsApprox})における仮定は妥当であると言える.

\subsection{異径・奥行方向距離を持つ場合の2粒子ホログラム}
\label{sec:appendix_deviation}
本論文では,記録面に対して垂直な方向,すなわち奥行方向の距離を持たない同径の近接粒子ホログラムを扱った.しかし,実際には当然奥行方向距離を有し粒子径の異なる粒子衝突が発生する.この場合,2粒子ホログラムはどのように変化するか,またこれらの条件でもスペクトル分布上の縞パターンによって近接検出が可能か調べる.

\subsubsection{奥行方向距離を持たない異径粒子組}
\label{sec:appendix_deviation_radius}
\ref{sec:twoParticleHologram}節と同様,座標原点中心の粒子1,$\psi_{A_1}$と$(\Delta \xi, \Delta \eta)$中心の粒子2,$\psi_{A_1}$について考える.粒子1の半径を$a_1$,粒子2の半径を$a_2 = a_1 + \Delta a$とする.これらについて,式(\ref{th:fourierOfA}),(\ref{th:particle1Fourier}),(\ref{th:particle2Fourier})より以下が成り立つ.
\begin{gather}
    \varphi_{A_1} (\alpha,\beta) = \mathcal{F} \left\{ \left|\psi_{A_1}\right|^2 \right\} = \cos{\left( \pi \lambda z_0 \gamma^2 \right)} \tilde{A}(\alpha,\beta;a_1) \\
    \varphi_{A_1} (\alpha,\beta) = \mathcal{F} \left\{ \left|\psi_{A_2}\right|^2 \right\} = \cos{ \left( \pi \lambda z_0 \gamma^2 \right) } \exp{\left( -2\pi \mathrm{j}\left( \alpha \Delta \xi + \beta \Delta \eta \right) \right)} \tilde{A}(\alpha,\beta;a_2) \\
    \tilde{A}(\alpha,\beta;a) = \mathcal{F}\left\{ A \right\} = \pi a^2 \frac{2J_1(2\pi a \gamma)}{2\pi a \gamma}
\end{gather}
したがって,原点以外におけるホログラムのフーリエ変換$\varphi(\alpha,\beta)$は以下で与えられる.
\begin{align}
    \varphi(\alpha,\beta) &= \varphi_{A_1}(\alpha,\beta) + \varphi_{A_2}(\alpha,\beta) \\
    &= \cos{\left( \pi \lambda z_0 \gamma^2 \right)} \left\{ \tilde{A}(\alpha,\beta;a_1) + \exp{\left( -2\pi \mathrm{j}\left( \alpha \Delta \xi + \beta \Delta \eta \right) \right)} \tilde{A}(\alpha,\beta;a_2) \right\}
\end{align}
スペクトル分布$S = |\varphi|^2$は以下のように計算できる.
\begin{align}
    S &= \left| \varphi(\alpha,\beta) \right|^2 \\
    &= \cos^2{\left( \pi \lambda z_0 \gamma^2 \right)} \left| \tilde{A}(\alpha,\beta;a_1) + \exp{\left( -2\pi \mathrm{j}\left( \alpha \Delta \xi + \beta \Delta \eta \right) \right)} \tilde{A}(\alpha,\beta;a_2) \right|^2 \\
    &= \cos^2{\left( \pi \lambda z_0 \gamma^2 \right)} \left\{ \tilde{A}_1^2+\tilde{A}_2^2 + 2\tilde{A}_1^2\tilde{A}_2^2 \cos{\left( 2\pi \left( \alpha \Delta \xi + \beta \Delta \eta \right) \right)} \right\}
    \label{appendixEq:diffdiamspec}
\end{align}
ただし,$\tilde{A}_1 = \tilde{A}(\alpha,\beta;a_1)$,$\tilde{A}_2 = \tilde{A}(\alpha,\beta;a_2)$とした.

$\tilde{A}_2$が以下の形で表されるとする.
\begin{gather}
    \tilde{A}_2 = \tilde{A}_1 + O(\gamma) \\
    O(\gamma) = (a+ \Delta a) \frac{J_1(2\pi (a + \Delta a) \gamma)}{\gamma} - a \frac{J_1(2\pi a \gamma)}{\gamma}
\end{gather}
このとき,式(\ref{appendixEq:diffdiamspec})は以下のようにも表せる.
\begin{equation}
    \frac{S}{4\cos^2{\left( \pi \lambda z_0 \gamma^2 \right)}} = \tilde{A}^2_1 \cos^2{\left( \pi \left( \alpha \Delta \xi + \beta \Delta \eta \right) \right)} +  \tilde{A}_1 O \cos^2{\left( \pi \left( \alpha \Delta \xi + \beta \Delta \eta \right) \right)} + O^2
    \label{appendixEq:residual}
\end{equation}
$O(\gamma)$は定義により動径$\gamma = \sqrt{\alpha^2 + \beta^2}$に依存する関数である.したがって,近接粒子組の半径が異なる場合もスペクトル分布上の縞パターン形状には影響を与えない.

\subsubsection{奥行方向距離を持つ同径粒子組}
\label{sec:appendix_deviation_z}

ここでも,座標原点中心の粒子1,$\psi_{A_1}$と$(\Delta \xi, \Delta \eta)$中心の粒子2,$\psi_{A_1}$について考える.粒子1の$z$軸座標を$z = 0$,粒子1と粒子2の奥行方向距離を$\Delta \zeta$とし,粒子1が粒子2よりも記録面に近い位置にあることとすると,記録面$z = z_1$における複素振幅$\psi(x,y;z_1)$は以下のように計算できる.
\begin{align}
    \psi(x,y; z = 0 - \delta z) &=  \mathcal{F}^{-1} \left\{ \mathcal{F}\left\{1 - A(x-\Delta \xi,y- \Delta \eta)\right\} \cdot G_{\Delta \zeta} \right\} \\
    \psi(x,y) &= \mathcal{F}^{-1} \left\{ \mathcal{F}\left\{ \psi(x,y;z = 0-\delta z) \cdot \left( 1 - A(x,y) \right) \right\} \cdot G_{z_1} \right\}
\end{align}
ここで,$G_z$は距離$z$の光伝播をあらわす角スペクトル法の伝搬関数である.粒子1と2の平面内距離が十分大きい場合は記録面の複素振幅は各粒子の物体光の和で表現できるが,今回のような近接条件では上式の計算が必要であり,またこの式の理論的な計算は困難である.そこで,本節では数値計算と実験によって$\Delta \zeta$とスペクトル分布上縞パターンの関係をみる.

ホログラムの数値生成及び実験による記録条件をTable \ref{table:appendixBcondition}に示す.また,実験装置の概要図をFig. \ref{fig:appendixBsetup}に示す.この実験装置は,Fig. \ref{fig:stripePatternExperiment}に示す装置から,マイクロメータによるガラスプレートの移動方向を$z$軸方向に変更したものである.

Fig. \ref{fig:appendixBresult}に,数値計算と実験それぞれによって得られたホログラムとそのスペクトル分布の例を示す.この例の奥行方向距離は$z=0$である.数値生成したホログラム,実験によって得たホログラムともに同様のパターンが記録されスペクトル上の縞パターンが得られていることがわかる.ここで得たスペクトル分布と,各$\Delta \zeta$におけるスペクトル分布の相互相関を計算したプロットをFig. \ref{fig:appendixBcorrelation}に示す.この相互相関は,$\Delta \zeta$が大きくなるにつれて減少していることがわかる.これは,$\Delta \zeta$が大きくなるにつれて,スペクトル分布の縞パターンが淡くなり確認が困難になるためである.数値計算・実験結果ともに相関係数の減少傾向は一致していることがわかる.それぞれのバイアスは異なるが,これは実験結果のスペクトル分布においてはもとのホログラムの背景ノイズによる高周波成分があるためである.

ホログラムの光伝播による像再生では,粒子像は奥行方向に$\Delta L \approx 4a^2/\lambda$ 程度の伸びを有することが知られている.本節における条件では伸び長さは$\Delta L \approx \SI{100}{\mm}$程度となるが,マイクロメータの可動域限界によりこの範囲までの計測は行えていない.今後は粒子伸び長さと粒子奥行方向距離に対する縞パターンの検出可能性を定量する必要がある.

\begin{table}[H]
    \centering
    \caption{Numerical generation and experimental recording conditions of holograms for $\Delta \zeta$ dependence experiment.}
    \label{table:appendixBcondition}
    \begin{tabular}{lll}
    Quantity                               & Value                                             & Unit     \\ \hline \hline
    Dot diameter $2a$                      & 250                                               & \si{\um} \\ \hline
    Dot distance in $x$-axis $\Delta \xi$  & 420                                               & \si{\um} \\ \hline
    Dot distance in $y$-axis $\Delta \eta$ & 50                                                & \si{\um} \\ \hline 
    Dot distance in $z$-axis $\Delta \zeta$& 0 to 3000 (numerical simulation)                  & \si{\mm} \\ 
                                           & 0 to 2780 in every 50 (experiment)                & \si{\mm} \\ \hline 
    Recorded wavelength $\lambda$          & 632.8                                             & \si{\nm} \\ \hline
    Propagated distance $z_0$              & 220                                               & \si{\mm} \\ \hline
    Pixel pitch of hologram $\Delta x$     & 10                                                & \si{\um}
    \end{tabular}
\end{table}


\begin{figure}[H]
    \centering
    \includegraphics[width=0.8\linewidth]{./Figure/7_Appendix/expdiagram.pdf}
    \caption{Schematic diagram of experimental setup. The $z$-axis distance between dots is varied by manipulating one of the two glass plates with dots printed on them with a micrometer. The planer distance between the two glass plates is fixed at $\Delta \xi = \SI{420}{\um}$,$\Delta \eta = \SI{50}{\um}$.}
    \label{fig:appendixBsetup}
\end{figure}


\begin{figure}[H]
    \centering
    \begin{subfigure}[t]{0.45\linewidth}
        \includegraphics[width=\linewidth]{./Figure/7_Appendix/data/holoexp.pdf}
        \caption{Positive hologram}
        \label{fig:appendixBresult:expholo}
    \end{subfigure}
    \hfill
    \begin{subfigure}[t]{0.45\linewidth}
        \includegraphics[width=\linewidth]{./Figure/7_Appendix/data/fftexp.pdf}
        \caption{Negative hologram}
        \label{fig:appendixBresult:expfft}
    \end{subfigure}

    \begin{subfigure}[t]{0.45\linewidth}
        \includegraphics[width=\linewidth]{./Figure/7_Appendix/data/holonum.pdf}
        \caption{Positive hologram spectrum}
        \label{fig:appendixBresult:numholo}
    \end{subfigure}
    \hfill
    \begin{subfigure}[t]{0.45\linewidth}
        \includegraphics[width=\linewidth]{./Figure/7_Appendix/data/fftnum.pdf}
        \caption{Negative hologram spectrum}
        \label{fig:appendixBresult:numfft}
    \end{subfigure}

    \caption{Example of recorded holograms and their spectra. The holograms are recorded at $\Delta \zeta = \SI{0}{\mm}$. The experimental hologram is cropped at the center so that the target fringe pattern is visible.} 
    \label{fig:appendixBresult}
\end{figure}

\begin{figure}[H]
    \centering
    \includegraphics[width=0.8\linewidth]{./Figure/7_Appendix/result.pdf}
    \caption{Cross-correlation between the spectrum of the hologram recorded at $\Delta \zeta = \SI{0}{\mm}$ and the spectra of the holograms recorded at $\Delta \zeta$. The range of $\Delta \zeta$ is given in Table \ref{table:appendixBcondition}.}
    \label{fig:appendixBcorrelation}
\end{figure}

\end{document}
